\documentclass[twoside,11pt]{article}

\usepackage{blindtext}

% Any additional packages needed should be included after jmlr2e.
% Note that jmlr2e.sty includes epsfig, amssymb, natbib and graphicx,
% and defines many common macros, such as 'proof' and 'example'.
%
% It also sets the bibliographystyle to plainnat; for more information on
% natbib citation styles, see the natbib documentation, a copy of which
% is archived at http://www.jmlr.org/format/natbib.pdf

% Available options for package jmlr2e are:
%
%   - abbrvbib : use abbrvnat for the bibliography style
%   - nohyperref : do not load the hyperref package
%   - preprint : remove JMLR specific information from the template,
%         useful for example for posting to preprint servers.
%
% Example of using the package with custom options:
%
% \usepackage[abbrvbib, preprint]{jmlr2e}

\usepackage{amsmath, amsthm, amssymb}
\usepackage{jmlr2e}

\usepackage{multirow}
\usepackage{makecell}
\usepackage{booktabs}
\usepackage{rotating}
\usepackage{tikz-cd}
\usepackage[utf8]{inputenc} % Enable UTF-8 encoding
\usepackage[T1]{fontenc}
\usepackage{textgreek}

\newtheorem{prop}{Proposition}
\newtheorem{thm}{Theorem}
\newtheorem{lem}{Lemma}
\newtheorem{assumption}{Assumption}

% Definitions of handy macros can go here

\newcommand{\dataset}{{\cal D}}
\newcommand{\fracpartial}[2]{\frac{\partial #1}{\partial  #2}}
\DeclareMathOperator*{\divm}{div}
\DeclareMathOperator*{\argmax}{arg\,max}
\DeclareMathOperator*{\argmin}{arg\,min}

% Heading arguments are {volume}{year}{pages}{date submitted}{date published}{paper id}{author-full-names}

\usepackage{lastpage}
\jmlrheading{23}{2025}{1-\pageref{LastPage}}{1/21; Revised 5/22}{9/22}{21-0000}{Author One and Author Two}

% Short headings should be running head and authors last names

\ShortHeadings{Probabilistic OVO}{Probabilistic OVO}
\firstpageno{1}

\begin{document}

\title{On change of priors in one-vs-one classification methods}

\author{\name Ondrej Šuch \email ondrejs@savbb.sk \\
       \addr Matematický ústav SAV\\
       Ďumbierska 1\\
       Banská Bystrica, 974 01, Slovakia
       \AND
       \name Peter Novotný \email peter.novotny@uniza.sk \\
       \addr Fakulta informatiky a riadenia\\
       Žilinská Univerzita v Žiline\\
       Žilina, 010 26, Slovakia
       \AND
       \name Ali Haidar \email haidar@savbb.sk \\
       \addr Matematický ústav SAV\\
       Ďumbierska 1\\
       Banská Bystrica, 974 01, Slovakia
       }

\editor{My editor}

\maketitle

\begin{abstract}%   <- trailing '%' for backward compatibility of .sty file
\blindtext
\end{abstract}

\begin{keywords}
  keyword one, keyword two, keyword three
\end{keywords}

\section{Introduction}

%\blindmathpaper

Multi-class classification is considered to be a more challenging problem than binary classification.  A natural approach to multi-class classification is to reduce it to a series of binary classification problems and deduce the multi-class decision by aggregating decisions for the binary problems. 

One popular method belonging to  this paradigm uses probabilistic modelling for one-vs-one series of binary problems. Notably, it is used as the basis for multi-class classification models using support vector machines (SVM) in LIBSVM library. However one can use the approach more generally,  because it can be applied to any probabilistic binary classification method. 

A key step in probabilistic one-vs-one modelling is the aggregation of results of individual binary classifiers - so-called \emph{coupling method}. The most commonly used coupling method is that proposed by Wu-Lin-Weng. As noted in [\cite{dohau}], the method is one of several non-canonical decisions used in multi-class SVM modelling. The goal of this paper is to compare its behavior with canonical methods. 


\input theory.tex

\input coupling.tex

\input experiment.tex

% Acknowledgements and Disclosure of Funding should go at the end, before appendices and references

\acks{All acknowledgements go at the end of the paper before appendices and references.
Moreover, you are required to declare funding (financial activities supporting the
submitted work) and competing interests (related financial activities outside the submitted work).
More information about this disclosure can be found on the JMLR website.}

% Manual newpage inserted to improve layout of sample file - not
% needed in general before appendices/bibliography.

\newpage

\input appendix.tex


\vskip 0.2in 
\bibliography{paper}

\end{document}
