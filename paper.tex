\documentclass[twoside,11pt]{article}

\usepackage{blindtext}

% Any additional packages needed should be included after jmlr2e.
% Note that jmlr2e.sty includes epsfig, amssymb, natbib and graphicx,
% and defines many common macros, such as 'proof' and 'example'.
%
% It also sets the bibliographystyle to plainnat; for more information on
% natbib citation styles, see the natbib documentation, a copy of which
% is archived at http://www.jmlr.org/format/natbib.pdf

% Available options for package jmlr2e are:
%
%   - abbrvbib : use abbrvnat for the bibliography style
%   - nohyperref : do not load the hyperref package
%   - preprint : remove JMLR specific information from the template,
%         useful for example for posting to preprint servers.
%
% Example of using the package with custom options:
%
% \usepackage[abbrvbib, preprint]{jmlr2e}

\usepackage{jmlr2e}
\usepackage{amsmath}

% Definitions of handy macros can go here

\newcommand{\dataset}{{\cal D}}
\newcommand{\fracpartial}[2]{\frac{\partial #1}{\partial  #2}}

% Heading arguments are {volume}{year}{pages}{date submitted}{date published}{paper id}{author-full-names}

\usepackage{lastpage}
\jmlrheading{23}{2022}{1-\pageref{LastPage}}{1/21; Revised 5/22}{9/22}{21-0000}{Author One and Author Two}

% Short headings should be running head and authors last names

\ShortHeadings{Sample JMLR Paper}{One and Two}
\firstpageno{1}

\begin{document}

\title{Comparison of canonical polychotomous coupling methods with the method of Wu-Lin-Weng}

\author{\name Ondrej Šuch \email ondrejs@savbb.sk \\
       \addr Matematický ústav SAV\\
       Ďumbierska 1\\
       Banská Bystrica, 974 01, Slovakia
       \AND
       \name Peter Novotný \email two@cs.berkeley.edu \\
       \addr Division of Computer Science\\
       University of California\\
       Berkeley, CA 94720-1776, USA}

\editor{My editor}

\maketitle

\begin{abstract}%   <- trailing '%' for backward compatibility of .sty file
\blindtext
\end{abstract}

\begin{keywords}
  keyword one, keyword two, keyword three
\end{keywords}

\section{Introduction}

%\blindmathpaper

Multi-class classification is considered to be a more challenging problem than binary classification.  A natural approach to multi-class classification is to reduce it to a series of binary classification problems and deduce the multi-class decision by aggregating decisions for the binary problems. 

One popular method belonging to  this paradigm uses probabilistic modelling for one-vs-one series of binary problems. Notably, it is used as the basis for multi-class classification models using support vector machines (SVM) in LIBSVM library. However one can use the approach more generally,  because it can be applied to any probabilistic binary classification method. 

A key step in probabilistic one-vs-one modelling is the aggregation of results of individual binary classifiers - so-called \emph{coupling method}. The most commonly used coupling method is that proposed by Wu-Lin-Weng. As noted in \cite{dohau}, the method is (one of) non-canonical decisions used in multi-class SVM modelling. The goal of this paper is to compare its behavior with canonical methods. 


\section{Theoretical basis of coupling methods}




Suppose that $K$ classes $C_1, \ldots, C_K$ are distributed according to probability distributions $p_i$ on a  space $X$. Let us denote the $r_{ij}(x)$ the output of Bayes (binary) classifier for $x$ in $X$ yielding the probability $p(x \in C_i \mid x\in C_i \cup C_j )$. Then we have
$$
r_{ij}(x)= \frac {p_i(x)}{p_i(x) + p_j(x)}.
$$
Suppose we have estimators $\hat r_{ij}$ for Bayes predictions $r_ij$ for all pairs $i\not= j$. A \emph{coupling method} obtains a multi-class probabilistic estimate $\boldsymbol{\hat p}= (\hat p_1, \ldots, \hat p_K)$ that satisfy
$$
\hat{r}_{ij}(x) \approx \frac {\hat p_i(x)}{\hat p_i(x) + \hat p_j(x)}. \label{eq:bt}
$$

Clearly, the equations in \eqref{eq:bt} can be transformed to a system of linear equations. One should note that 
\begin{itemize}
\item the resulting system of equations will usually not be consistent, because $\hat{r}_{ij}$ are only estimates of true values $r_{ij}$,
\item when probabilities are parametrized in other ways (e.g. other common parametrizations are as odds, or log-odds), the system of equations \eqref{eq:bt} is non-linear.
\end{itemize}


Here is a citation \cite{chow:68}.

% Acknowledgements and Disclosure of Funding should go at the end, before appendices and references

\acks{All acknowledgements go at the end of the paper before appendices and references.
Moreover, you are required to declare funding (financial activities supporting the
submitted work) and competing interests (related financial activities outside the submitted work).
More information about this disclosure can be found on the JMLR website.}

% Manual newpage inserted to improve layout of sample file - not
% needed in general before appendices/bibliography.

\newpage

\appendix
\section{}
\label{app:theorem}

% Note: in this sample, the section number is hard-coded in. Following
% proper LaTeX conventions, it should properly be coded as a reference:

%In this appendix we prove the following theorem from
%Section~\ref{sec:textree-generalization}:

In this appendix we prove the following theorem from
Section~6.2:

\noindent
{\bf Theorem} {\it Let $u,v,w$ be discrete variables such that $v, w$ do
not co-occur with $u$ (i.e., $u\neq0\;\Rightarrow \;v=w=0$ in a given
dataset $\dataset$). Let $N_{v0},N_{w0}$ be the number of data points for
which $v=0, w=0$ respectively, and let $I_{uv},I_{uw}$ be the
respective empirical mutual information values based on the sample
$\dataset$. Then
\[
	N_{v0} \;>\; N_{w0}\;\;\Rightarrow\;\;I_{uv} \;\leq\;I_{uw}
\]
with equality only if $u$ is identically 0.} \hfill\BlackBox

\section{}

\noindent
{\bf Proof}. We use the notation:
\[
P_v(i) \;=\;\frac{N_v^i}{N},\;\;\;i \neq 0;\;\;\;
P_{v0}\;\equiv\;P_v(0)\; = \;1 - \sum_{i\neq 0}P_v(i).
\]
These values represent the (empirical) probabilities of $v$
taking value $i\neq 0$ and 0 respectively.  Entropies will be denoted
by $H$. We aim to show that $\fracpartial{I_{uv}}{P_{v0}} < 0$....\\

{\noindent \em Remainder omitted in this sample. See http://www.jmlr.org/papers/ for full paper.}


\vskip 0.2in
\bibliography{sample}

\end{document}
