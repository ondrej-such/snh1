\section{Introduction}

%\blindmathpaper


Distribution shift is an important problem facing machine-learning practitioners 
\cite{zhang2023dive, sugiyama2007covariate}.  Using probabilistic formulation, we say the distribution shift occurs if there is a change  in the joint probability distribution $p(X,Y)$ of covariates $X$ with the dependent variable $Y$  between training and testing. Noting the factorizations 
$$
p(X,Y) = p(Y\mid X)p(X) = p(X \mid Y )p(Y)
$$
one may distinguish the following subcategories of distribution shift:
\begin{itemize}
	\item covariate shift, if $p(X)$ changes, but $p(Y\mid X)$ is unchanged,
	\item label shift, if $p(Y)$ changes, but $p(X \mid Y)$ remains the same,
	\item concept shift, if $p(Y \mid X)$ itself changes.
\end{itemize}

We will be concerned with label shift in classification problems, where $p(Y)$ 
is just the prior on classes.  Some aspects of label shift were studied in
earlier works including test time adaptation
\cite{vsipka2022hitchhiker}, shift detection and quantification 
\cite{lipton2018detecting}, 
statistical properties of label shift estimators \cite{garg2020unified}, 
and Fisher consistency of estimators of priors \cite{tasche2017fisher}. 
Let us consider  the question 
how to adapt a classifier when the  prior is updated.

To provide a concrete example, consider a driver assistance system recognizing traffic signs. 
Suppose a speed limit sign was maliciously altered to indicate 80mph limit instead of 30mph.
To avoid being fooled by the alteration, the system could utilize priors.
The distribution of traffic signs will change markedly when the car leaves highway and enters streets in a city.
It is very unlikely to encounter 80mph limit in a city setting. By taking 
the city priors  into account, the system would be more likely to make the proper classification. 

Some systems are unable to incorporate new priors without rebuilding 
the classifier, e.g., those based 
on fuzzy logic or some tree methods. 
However, change of priors can be effectively dealt with in probabilistic 
classifiers. The basic idea is that Bayes theorem stipulates \emph{exactly} 
what should happen  to the prediction of the  Bayes classifier when priors 
change (see \eqref{eq:changePrior}).  In practice we assume that a trained 
classifier closely approximates the Bayes classifier. Therefore we can make 
adjustment to our prediction by applying the same transformation of posterior 
as would happen with the Bayes classifier. 

There is a subtle issue with this approach  when handling probabilistic 
classifiers arising in \emph{probabilistic} one-vs-one classification framework.   
One-vs-one classification is a two-step framework to solve 
multiclass classification problems.
In the first step it reduces the multiclass problem to a series of binary
classification problems. 
In the second step it deduces the multiclass decision by aggregating decisions 
for the binary problems. Since the procedure is two-step, there is a possibility 
to apply the adjustment of priors either to the probabilistic distributions 
obtained after the first step, or after the second step. It is not clear what 
should be the preferred course of action and this problem motivates our paper. We note that the dilemma does not appear
in \emph{non-probabilistic} one-vs-one classification, e.g., when voting is used in the second step \cite{galar2011overview}.


Our approach to the problem is to evaluate practical performance of different 
methods to aggregate probabilistic binary predictions. Such methods---called coupling methods---first appeared in work of \cite{refregier1991probabilistic} when making 
multi-class classification using neural networks. 
Many authors followed with alternative proposals 
\cite{price1994pairwise, hastie1998classification,  zahorian1999partitioned,
wu2004probability, vsuch2015new, vsuch2016bayes}.

We formulate the underlying principles of one-vs-one probabilistic classification in Section 2. In Section 3 we present  different classes of coupling methods. 
We pay special attention to methods that commute with change of priors---Bayes covariant methods, 
since such methods provide an  elegant solution to our problem. 
We introduce new coupling methods by forming weighted ensembles. We prove such ensemble of Bayes covariant methods is again Bayes covariant
when a normalizing condition on weights holds (Proposition \ref{prop:ensemble}).  
Our experiments are described in Section 4. In Section 5 we summarize our findings, 
provide practical guidance and outline future research directions.