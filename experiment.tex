\section{Methodology.}

\subsection{Datasets}

It is natural to consider the benchmark datasets used in previous evaluation of coupling methods [\cite{wu2004probability}]. The authors considered altogether seven datasets with number of classes ranging from 3 to 26. For each dataset they extracted 40 pairs of training and testing datasets. Half of these were smaller (300 training samples and 500 testing samples) and the other half was larger (800 training samples and 1000 testing samples). In our experiments we use only the larger pairs. 

As a preliminary step we conducted analysis of linear separability of the datasets. We used ECOS solver in CVXR library to solve the underlying quadratic program. The results are reported in Table \ref{tab:sep}.

% latex table generated in R 4.5.1 by xtable 1.8-4 package
% Wed Aug 13 15:27:15 2025
\begin{table}[ht]
\centering
\begin{tabular}{lrr}
  \hline
Dataset & $K$ & Percentage linearly separable \\ 
  \hline
dna &    3 & 100 \%    \\ 
  letter &   26 & 99 \%    \\ 
  mnist &   10 & 100 \%    \\ 
  satimage &    6 & 84 \%    \\ 
  segment &    7 & 89 \%    \\ 
  usps &   10 & 100 \%    \\ 
  waveform &    3 & 0 \%    \\ 
   \hline
\end{tabular}
\caption{The benchmark datasets with indication of the number of classes $K$ and the percentage of binary classification problems that are linearly separable} 
\label{tab:sep}
\end{table}


We can infer that except for 'waveform' dataset, majority of binary problems are linearly separable.

\subsection{Classification methodology}

We have opted to use LDA as a binary classification tool. Its appeal is that it is applicable even for linearly separable datasets, thus obviating additional crossvalidation step incurred by methods that use regularization (e.g. penalized logistic regression or SVM). It usually provides results close to logistic regression [\cite{james2013introduction}].

For preprocessing we first removed features that were constant across classes. Then we projected the data using PCA to the subspace for which corresponding singular values were $> tol$, where $tol$ is a hyper-parameter of an experiment. We used PCA without scaling or centering since the datasets' features were already normalized to interval $[-1,1]$. 

For training LDA we used MASS library in R. We implemented coupling methods in C++ using Eigen library. As arguments to the coupling methods we use log-odds, i.e. instead of $r_{ij}$ they expect
$$
\tilde r_{ij} = \log \biggl(\frac{1}{r_{ji}} - 1\biggr).
$$
The resulting matrix $\tilde{\boldsymbol{R}}$ is skew symmetric (with the convention of having zeroes on the diagonal).

This parametrization allows for precise representation of extremely small probabilities which are sometimes produced by LDA. These small probabilities might be expected given that a large fraction of binary problems is linearly separable. 

Small probabilities generally have very little effect on the implementation of Bayes covariant methods which generally operate using log-odds. However small probabilities do affect both the method of Wu-Lin-Weng as well sa radial method, because then Assumption \ref{ass:1} fails to hold. 

%\begin{figure}[!ht]
%	\includegraphics[width = 0.5\linewidth]{graph/dna-lda.pdf}
%	\includegraphics[width = 0.5\linewidth]{graph/waveform-lda.pdf}
%	\caption{Projections of three-class datasets using LDA Left: dna, right: waveform}
%\end{figure}

\section{Experimental results.}

The choice of a coupling method is an important task for practitioners. We aim to help by posing the following interrelated questions 

\begin{itemize}
\item[a)] Which non-parametric coupling methods shows good performance across diverse data sets? Is it the orthodox method of Wu-Lin-Weng?
\item[b)] If the answer to a) is a coupling method which is not Bayes covariant, can we find a parametric method Bayes covariant method which  matches its performance?
\item[c)] If the answer to b) is negative, is it better to apply change of priors before or after coupling? Additionally, how serious is the impact of the lack of Bayes covariance in practice?
\end{itemize}

These questions will be put to test in Sections \ref{sec:exp1}, \ref{sec:exp2}, and \ref{sec:exp3}.



\subsection{Experiment: Accuracy of canonical coupling methods on benchmark datasets} \label{sec:exp1}

The first experiment compares the performance of three parameterless coupling methods
\begin{itemize}
\item method of Wu-Lin-Weng, which is widely used in software libraries, and thus can be viewed as an orthodox choice,
\item normal coupling, which is a Bayes covariant method,
\item radial coupling, which is not Bayes covariant.
\end{itemize}

Although previous evaluations of the latter two methods were done in [\cite{vsuch2016bayes}], they used phonetic data for which ground truth may be imprecise. 

% % latex table generated in R 4.4.1 by xtable 1.8-4 package
% Sun Apr  6 13:12:57 2025
\begin{table}[ht]
\centering
\begin{tabular}{lrrr}
  \hline
dataset & normal & radial & Wu-Lin-Weng \\ 
  \hline
dna & 0.81 & 0.84 & 0.84 \\ 
  letter & 0.35 & 0.75 & 0.77 \\ 
  mnist & 0.48 & 0.47 & 0.50 \\ 
  satimage & 0.68 & 0.81 & 0.81 \\ 
  segment & 0.57 & 0.93 & 0.94 \\ 
  usps & 0.22 & 0.48 & 0.78 \\ 
  waveform & 0.86 & 0.86 & 0.86 \\ 
   \hline
\end{tabular}
\caption{Comparison of accuracy for three parameterless coupling methods} 
\label{tab:multi}
\end{table}


From Figure \ref{fig:exp1-plot1} we can clearly see that Wu-Lin-Weng's method has the best performance. On some datasets it trounces normal coupling method by a wide margin. The edge over the radial method is smaller with only mnist and usps datasets showing a significant difference for smaller values of $tol$ hyperparameter. 

\begin{figure}
	\includegraphics{graphs/exp1-plot1.pdf}
	\caption{Success rate plotted against the logarithm of $tol$ hyper-parameter, evaluated for different datasets.}
	\label{fig:exp1-plot1}
\end{figure}

\subsection{Experiment: Accuracy of Bayes covariant methods without parameters}
 \label{sec:exp2}

The method of Wu-Lin-Weng provides leading performance among the canonical {nonparametric} methods examined in the previous experiment. However in view of Theorem \ref{thm:K3} the Wu-Lin-Weng's method is not Bayes covariant. 

In this experiment we investigated whether we can (always) find  a (possibly \emph{parametric}) Bayes covariant coupling method that could match the performance of Wu-Lin-Weng's method. We investigate  the case of three class datasets. This is a natural starting point, since it is the smallest number of classes where the question is meaningful to asks. All experiments use value of 0.05 for $tol$ hyperparameter based on results of Section \ref{sec:exp1}. Our experiment has three parts. 


In the first part, we consider whether four parameterless Bayes covariant method could match the performance of Wu-Lin-Weng method. In our comparison we include the normal method, as well as the three arboreal methods $ \boldsymbol{s}_1,~ \boldsymbol{s}_2,~ \boldsymbol{s}_3$ that are not canonical. We consider all possible triples of classes in all datasets. The resulting statistics are shown in Table \ref{tab:step2}. 

% latex table generated in R 4.5.1 by xtable 1.8-4 package
% Thu Aug 14 09:38:32 2025
\begin{table}[ht]
\centering
\begin{tabular}{lr}
  \hline
Outcome & Percentage \\ 
  \hline
 $\boldsymbol{e}_3$ at least as accurate as $\boldsymbol{wlw}$ & 24 \%   \\ 
  one of $\boldsymbol{s}_1, \boldsymbol{s}_2, \boldsymbol{s}_3$ at least as accurate as $\boldsymbol{wlw}$ & 53 \%   \\ 
  $\boldsymbol{wlw}$ more accurate than $\boldsymbol{e}_3, \boldsymbol{s}_1, \boldsymbol{s}_2, \boldsymbol{s}_3$ & 39 \%   \\ 
   \hline
\end{tabular}
\caption{Comparison of Wu--Lin--Weng's method $\boldsymbol{wlw}$ and  parameterless coupling methods on three class subsets. } 
\label{tab:step2}
\end{table}


We can see that in slightly less than half of the cases none of the four parameterless Bayes covariant methods was able to match the performance of Wu-Lin-Weng's method. Therefore we will extend search to Bayes covariant methods with trainable parameters. 


\subsection{Experiment: A parametric family}

Let us define two parameter family of Bayes covariant coupling methods
\begin{align}
a_1 \boldsymbol{s}_1 \oplus a_2 \boldsymbol{s}_2 \oplus a_3 \boldsymbol{s}_3,\quad\textrm{where } a_1 + a_2 +a_3 = 1. \label{eq:family}
\end{align}
Note that this family includes the normal coupling method as well as the three arboreal methods $\boldsymbol{s}_1, \boldsymbol{s}_2, \boldsymbol{s}_3$.

In this experiment we  conduct a grid search for methods matching performance of Wu-Lin-Weng method. Grid search works best in low dimension, when restricted to a compact subset. To that end we restrict our search to the simplex formed with nonnegative coefficients $a_i \geq 0$, which form a compact subset.  To reduce computational burden we selected 1000 random chosen three-class subproblems from those where Wu-Lin-Weng's method outperformed normal coupling. The summary statistics are shown in Figure \ref{fig:par-bc}.

\begin{figure}[!ht]
\includegraphics{graphs/exp2-summary.pdf}
\caption{Comparison of success rate for parametric Bayes covariant methods \eqref{eq:family} vs. Wu-Lin-Weng's method}
\label{fig:par-bc}
\end{figure}

We can see that only in minority of cases we found a parametric Bayes covariant coupling method matching the performance of Wu-Lin-Weng's method. This finding comes with caveat that we searched only in the simplex $a_i \geq 0$. 

\subsection{Example: }

We determined that among the 1000 three-class problems we have examined, the largest gap between Bayes-covariant parametric methods and Wu-Lin-Weng method occurred for mnist dataset and classes 5,6,9. Figure 

\begin{itemize}
	\item We searched within a much larger rectangle $D$ containing the simplex $a_i\geq 0$, which in this case is a triangle $T$.
	\item We used Brier score as a smooth proxy for measuring the quality of multi-class classifier. This score is much smoother than 0-1 score which measures accuracy.
\end{itemize}

The results are shown in Figure \ref{fig:score}


\begin{figure}[!ht]
	\includegraphics{graphs/exp2-detail.pdf}
	\caption{Color map indicates the value of Brier score computed for an ensemble of coupling methods on the testing dataset. The vertices triangle indicate the arboreal coupling methods, and the black circle the location of the optimum.}
	\label{fig:score}
\end{figure}

We can see that indeed the optimum over the rectangle lies outside of the simplex. Let us compare the resulting accuracy of the optima in $\Delta$ and $D$. The results are shown in Table \ref{tab:step5}.

\input tab-step5.tex

\subsection{Experiment 3: Effect of change of priors}  \label{sec:exp3}
%\subsection{Wu-Lin-Weng's method}



