\appendix

\section{}
\label{app:bc1}

In this appendix we provide proofs of statements used in Section \label{sec:bc2}.

\begin{prop} \label{prop:bcprop}
	Let $\boldsymbol{v}$ be a coupling method. If Assumption \ref{ass:1} holds, the following statements are equivalent:
	\begin{itemize}
		\item[a)] The coupling method $\boldsymbol{v}$ is Bayes covariant.
		\item[b)] The coupling method $\boldsymbol{v}$ satisfies $BC_{ij}$ for any pair of integers  $i\not= j$ with $1\leq i,j \leq K$.
		\item[c)] The coupling method $\boldsymbol{v}$ satisfies $BC_{12}, BC_{23}, \ldots, BC_{K-1,K}$.
	\end{itemize}
\end{prop}
\begin{proof}
It follows directly from the definition of $BC_{ij}$ that a) implies b). Also, clearly b) implies c). If the assumptions of c) hold, then the vectors $\boldsymbol{v}(\boldsymbol{R})^\pi$ and $ \boldsymbol{v}(\boldsymbol{R}^\pi)$ are proportional, but since both have components adding up to one, it follows they are equal and thus a) holds.
\end{proof}

\begin{lem} \label{lem:transitivity}
	Suppose Assumption \ref{ass:1} holds. If a coupling method $\boldsymbol{v}$ satisfies $BC_{ij}$ and $BC_{jk}$ then it satisfies $BC_{ik}$.
\end{lem}
\begin{proof}
	Let $\pi$ be a change of priors. From the assumptions we have 
	\begin{align}
		\alpha_{ij}(\boldsymbol{v}(\boldsymbol{R})^\pi ) &= \alpha_{ij}(\boldsymbol{v}(\boldsymbol{R}^\pi) ) \\
		\alpha_{jk}(\boldsymbol{v}(\boldsymbol{R})^\pi ) &= \alpha_{jk}(\boldsymbol{v}(\boldsymbol{R}^\pi) )
	\end{align}
	Since $\alpha_{ik}(\boldsymbol{p}) = \alpha_{ij}(\boldsymbol{p})  \alpha_{jk}(\boldsymbol{p})$ it follows that for any $\boldsymbol{R}$
	\begin{align}
		\alpha_{ik}(\boldsymbol{v}(\boldsymbol{R})^\pi ) &= \alpha_{ik}(\boldsymbol{v}(\boldsymbol{R}^\pi) )
	\end{align}
	which means that the coupling method satisfies $BC_{ik}$.
\end{proof}

The following proposition restricts the set of Bayes covariant coupling methods for $K=3$.

\begin{prop} \label{prop:bc3}
	For $K=3$ any Bayes covariant coupling method is uniquely determined by its prediction for  matrices of the form
	$$
	\boldsymbol{R_s} = \begin{pmatrix} \cdot & s & 1 -s \\  1-s & \cdot & s \\ s & 1-s & \cdot \end{pmatrix}.
	$$
\end{prop}
\begin{proof}
	Suppose $\boldsymbol{v}$ is a coupling method,  the priors are $\Pi = (P_1, P_2,P_3)$, and the binary classifiers yield prediction matrix 
	\begin{align}
		\boldsymbol{R} = \begin{pmatrix} \cdot & r_{12} & 1 - r_{31} \\ 1-r_{12} & \cdot & r_{23} \\
			r_{31} & 1- r_{23} & \cdot \end{pmatrix}.
	\end{align}
	
	Suppose there was a change of priors $\pi$ such that $\boldsymbol{R}^\pi$ would be belong to the one-parameter family $\boldsymbol{R}_s$ with $s$ in $(0,1)$. Then we have
	$$
	\boldsymbol{v}(\boldsymbol{R})^\pi = \boldsymbol{v}(\boldsymbol{R}^\pi) = \boldsymbol{v}(\boldsymbol{R}_s).
	$$
	Inverting action of $\pi$ would then determine $\boldsymbol{v}(\boldsymbol{R})$.
	
	In order to construct such $\pi$ let us introduce a new parametrization of the matrix of binary  predictions
	\begin{align}
		\begin{split}
			q_{12} &= \frac{1}{r_{12}} -1 \\
			q_{23} &= \frac{1}{r_{23}} -1 \\
			q_{31} &= \frac{1}{r_{31}} -1 
		\end{split}
	\end{align}
	Under the change of priors $\pi:\Pi \rightarrow \Pi' = (P'_1, P'_2, P'_3)$ we have
	\begin{align}
		q'_{ij}	= q_{ij} \cdot \frac{P_i'}{P_i} \frac{P_j}{P'_j}.
	\end{align}
	One checks easily that $\Pi' \propto (1, \frac{q_{12}}{q_{23}}, \frac{q_{12}}{q_{31}})$ yields the desired change of priors.
	
\end{proof}


%\section{}
%\label{app:explicit}
%
%% Note: in this sample, the section number is hard-coded in. Following
%% proper LaTeX conventions, it should properly be coded as a reference:
%
%%In this appendix we prove the following theorem from
%%Section~\ref{sec:textree-generalization}:
%
%In this appendix we provide explicit predictions of coupling methods used in the paper.
%
%
%
%In this appendix we prove the following theorem from
%Section~6.2:
%
%\noindent
%{\bf Theorem} {\it Let $u,v,w$ be discrete variables such that $v, w$ do
%	not co-occur with $u$ (i.e., $u\neq0\;\Rightarrow \;v=w=0$ in a given
%	dataset $\dataset$). Let $N_{v0},N_{w0}$ be the number of data points for
%	which $v=0, w=0$ respectively, and let $I_{uv},I_{uw}$ be the
%	respective empirical mutual information values based on the sample
%	$\dataset$. Then
%	\[
%	N_{v0} \;>\; N_{w0}\;\;\Rightarrow\;\;I_{uv} \;\leq\;I_{uw}
%	\]
%	with equality only if $u$ is identically 0.} \hfill\BlackBox
%
%\section{}
%
%\noindent
%{\bf Proof}. We use the notation:
%\[
%P_v(i) \;=\;\frac{N_v^i}{N},\;\;\;i \neq 0;\;\;\;
%P_{v0}\;\equiv\;P_v(0)\; = \;1 - \sum_{i\neq 0}P_v(i).
%\]
%These values represent the (empirical) probabilities of $v$
%taking value $i\neq 0$ and 0 respectively.  Entropies will be denoted
%by $H$. We aim to show that $\fracpartial{I_{uv}}{P_{v0}} < 0$....\\
%
%{\noindent \em Remainder omitted in this sample. See http://www.jmlr.org/papers/ for full paper.}