\appendix

\section{}
\label{app:bc1}

In this appendix we provide proofs of statements used in Section \label{sec:bc2}.

\begin{prop}
	Let $\boldsymbol{v}$ be a coupling method. The following statements are equivalent:
	\begin{itemize}
		\item[a)] The coupling method $\boldsymbol{v}$ is Bayes covariant.
		\item[b)] The coupling method $\boldsymbol{v}$ satisfies $BC_{12}, BC_{23}, \ldots, BC_{K-1,K}$.
		\item[c)] The coupling method $\boldsymbol{v}$ satisfies $BC_{ij}$ for any pair of integers between 1 and $K$ with $i\not= j$.
	\end{itemize}
\end{prop}
\begin{proof}
	
\end{proof}

\begin{lem} \label{lem:transitivity}
	If a coupling method $\boldsymbol{v}$ satisfies $BC_{ij}$ and $BC_{jk}$ then it satisfies $BC_{ik}$.
\end{lem}
\begin{proof}
	Let $\pi$ be a change of priors. From the assumptions we have 
	\begin{align}
		\alpha_{ij}(\boldsymbol{v}(\boldsymbol{R})^\pi ) &= \alpha_{ij}(\boldsymbol{v}(\boldsymbol{R}^\pi) ) \\
		\alpha_{jk}(\boldsymbol{v}(\boldsymbol{R})^\pi ) &= \alpha_{jk}(\boldsymbol{v}(\boldsymbol{R}^\pi) )
	\end{align}
	Since $\alpha_{ik}(\boldsymbol{p}) = \alpha_{ij}(\boldsymbol{p})  \alpha_{jk}(\boldsymbol{p})$ it follows that for any $\boldsymbol{R}$
	\begin{align}
		\alpha_{ik}(\boldsymbol{v}(\boldsymbol{R})^\pi ) &= \alpha_{ik}(\boldsymbol{v}(\boldsymbol{R}^\pi) )
	\end{align}
	which means that the coupling method satisfies $BC_{ik}$.
\end{proof}


\section{}
\label{app:explicit}

% Note: in this sample, the section number is hard-coded in. Following
% proper LaTeX conventions, it should properly be coded as a reference:

%In this appendix we prove the following theorem from
%Section~\ref{sec:textree-generalization}:

In this appendix we provide explicit predictions of coupling methods used in the paper.



In this appendix we prove the following theorem from
Section~6.2:

\noindent
{\bf Theorem} {\it Let $u,v,w$ be discrete variables such that $v, w$ do
	not co-occur with $u$ (i.e., $u\neq0\;\Rightarrow \;v=w=0$ in a given
	dataset $\dataset$). Let $N_{v0},N_{w0}$ be the number of data points for
	which $v=0, w=0$ respectively, and let $I_{uv},I_{uw}$ be the
	respective empirical mutual information values based on the sample
	$\dataset$. Then
	\[
	N_{v0} \;>\; N_{w0}\;\;\Rightarrow\;\;I_{uv} \;\leq\;I_{uw}
	\]
	with equality only if $u$ is identically 0.} \hfill\BlackBox

\section{}

\noindent
{\bf Proof}. We use the notation:
\[
P_v(i) \;=\;\frac{N_v^i}{N},\;\;\;i \neq 0;\;\;\;
P_{v0}\;\equiv\;P_v(0)\; = \;1 - \sum_{i\neq 0}P_v(i).
\]
These values represent the (empirical) probabilities of $v$
taking value $i\neq 0$ and 0 respectively.  Entropies will be denoted
by $H$. We aim to show that $\fracpartial{I_{uv}}{P_{v0}} < 0$....\\

{\noindent \em Remainder omitted in this sample. See http://www.jmlr.org/papers/ for full paper.}