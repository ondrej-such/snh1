%Version 3.1 December 2024
% See section 11 of the User Manual for version history
%
%%%%%%%%%%%%%%%%%%%%%%%%%%%%%%%%%%%%%%%%%%%%%%%%%%%%%%%%%%%%%%%%%%%%%%
%%                                                                 %%
%% Please do not use \input{...} to include other tex files.       %%
%% Submit your LaTeX manuscript as one .tex document.              %%
%%                                                                 %%
%% All additional figures and files should be attached             %%
%% separately and not embedded in the \TeX\ document itself.       %%
%%                                                                 %%
%%%%%%%%%%%%%%%%%%%%%%%%%%%%%%%%%%%%%%%%%%%%%%%%%%%%%%%%%%%%%%%%%%%%%

%%\documentclass[referee,sn-basic]{sn-jnl}% referee option is meant for double line spacing

%%=======================================================%%
%% to print line numbers in the margin use lineno option %%
%%=======================================================%%

%%\documentclass[lineno,pdflatex,sn-basic]{sn-jnl}% Basic Springer Nature Reference Style/Chemistry Reference Style

%%=========================================================================================%%
%% the documentclass is set to pdflatex as default. You can delete it if not appropriate.  %%
%%=========================================================================================%%

%%\documentclass[sn-basic]{sn-jnl}% Basic Springer Nature Reference Style/Chemistry Reference Style

%%Note: the following reference styles support Namedate and Numbered referencing. By default the style follows the most common style. To switch between the options you can add or remove ?Numbered? in the optional parenthesis. 
%%The option is available for: sn-basic.bst, sn-chicago.bst%  
 
%%\documentclass[pdflatex,sn-nature]{sn-jnl}% Style for submissions to Nature Portfolio journals
%%\documentclass[pdflatex,sn-basic]{sn-jnl}% Basic Springer Nature Reference Style/Chemistry Reference Style
\documentclass[pdflatex,sn-mathphys-num]{sn-jnl}% Math and Physical Sciences Numbered Reference Style
%%\documentclass[pdflatex,sn-mathphys-ay]{sn-jnl}% Math and Physical Sciences Author Year Reference Style
%%\documentclass[pdflatex,sn-aps]{sn-jnl}% American Physical Society (APS) Reference Style
%%\documentclass[pdflatex,sn-vancouver-num]{sn-jnl}% Vancouver Numbered Reference Style
%%\documentclass[pdflatex,sn-vancouver-ay]{sn-jnl}% Vancouver Author Year Reference Style
%%\documentclass[pdflatex,sn-apa]{sn-jnl}% APA Reference Style
%%\documentclass[pdflatex,sn-chicago]{sn-jnl}% Chicago-based Humanities Reference Style

%%%% Standard Packages
%%<additional latex packages if required can be included here>

\usepackage{graphicx}%
\usepackage{multirow}%
\usepackage{amsmath,amssymb,amsfonts}%
\usepackage{amsthm}%
\usepackage{mathrsfs}%
\usepackage[title]{appendix}%
\usepackage{xcolor}%
\usepackage{textcomp}%
\usepackage{manyfoot}%
\usepackage{booktabs}%
\usepackage{algorithm}%
\usepackage{algorithmicx}%
\usepackage{algpseudocode}%
\usepackage{listings}%
%%%%

%%%%%=============================================================================%%%%
%%%%  Remarks: This template is provided to aid authors with the preparation
%%%%  of original research articles intended for submission to journals published 
%%%%  by Springer Nature. The guidance has been prepared in partnership with 
%%%%  production teams to conform to Springer Nature technical requirements. 
%%%%  Editorial and presentation requirements differ among journal portfolios and 
%%%%  research disciplines. You may find sections in this template are irrelevant 
%%%%  to your work and are empowered to omit any such section if allowed by the 
%%%%  journal you intend to submit to. The submission guidelines and policies 
%%%%  of the journal take precedence. A detailed User Manual is available in the 
%%%%  template package for technical guidance.
%%%%%=============================================================================%%%%


\usepackage{makecell}
\usepackage{booktabs}
\usepackage{rotating}
\usepackage{tikz-cd}
\usepackage[utf8]{inputenc} % Enable UTF-8 encoding
\usepackage[T1]{fontenc}
\usepackage{textgreek}
\usepackage{ifthen}
%\undef\orcidlogo


\newtheorem{prop}{Proposition}
\newtheorem{thm}{Theorem}
\newtheorem{lem}{Lemma}
\newtheorem{assumption}{Assumption}

% Definitions of handy macros can go here

\newcommand{\dataset}{{\cal D}}
\newcommand{\fracpartial}[2]{\frac{\partial #1}{\partial  #2}}
\DeclareMathOperator*{\divm}{div}
\DeclareMathOperator*{\argmax}{arg\,max}
\DeclareMathOperator*{\argmin}{arg\,min}




%% as per the requirement new theorem styles can be included as shown below
\theoremstyle{thmstyleone}%
\newtheorem{theorem}{Theorem}%  meant for continuous numbers
%%\newtheorem{theorem}{Theorem}[section]% meant for sectionwise numbers
%% optional argument [theorem] produces theorem numbering sequence instead of independent numbers for Proposition
\newtheorem{proposition}[theorem]{Proposition}% 
%%\newtheorem{proposition}{Proposition}% to get separate numbers for theorem and proposition etc.

\theoremstyle{thmstyletwo}%
\newtheorem{example}{Example}%
\newtheorem{remark}{Remark}%

\theoremstyle{thmstylethree}%
\newtheorem{definition}{Definition}%

\raggedbottom
%%\unnumbered% uncomment this for unnumbered level heads

\usepackage{etoolbox}
\let\orcidlogo\relax
\usepackage{orcidlink}

\begin{document}

\title[Change of priors]{New Bayes covariant coupling methods for adapting to change of priors in one-vs-one classification}

%%=============================================================%%
%% GivenName	-> \fnm{Joergen W.}
%% Particle	-> \spfx{van der} -> surname prefix
%% FamilyName	-> \sur{Ploeg}
%% Suffix	-> \sfx{IV}
%% \author*[1,2]{\fnm{Joergen W.} \spfx{van der} \sur{Ploeg} 
%%  \sfx{IV}}\email{iauthor@gmail.com}
%%=============================================================%%

\author*[1]{\fnm{Ondrej} \sur{Šuch} \orcidlink{0000-0002-8140-241X}}\email{ondrejs@savbb.sk}
%\equalcont{These authors contributed equally to this work.}

\author[2]{\fnm{Peter} \sur{Novotný}\orcidlink{0000-0002-7949-5215}}\email{peter.novotny@fri.uniza.sk}

\author[1,3]{\fnm{Ali} \sur{Haidar}\orcidlink{0000-0002-7498-4220}}\email{haidar@savbb.sk}
%\equalcont{These authors contributed equally to this work.}


\affil[1]{\orgdiv{Matematický ústav}, \orgname{Slovenská akadémia vied}, 
\orgaddress{\street{Ďumbierska 1}, \city{Banská Bystrica}, \postcode{97411}, \country{Slovakia}}}

\affil[2]{\orgdiv{Fakulta riadenia a informatiky}, \orgname{Žilinská Univerzita v Žiline}, 
\orgaddress{\street{Univerzitná 8215/6}, \city{Žilina}, \postcode{01026},  \country{Slovakia}}}

\affil[3]{\orgdiv{Fakulta matematiky, fyziky a informatiky}, 
\orgname{Univerzita Komenského}, \orgaddress{\street{Mlynská dolina F1}, \city{Bratislava}, \postcode{84248},  
\country{Slovakia}}}

%%==================================%%
%% Sample for unstructured abstract %%
%%==================================%%

\abstract{Label shift is one kind of distribution shift affecting classification models. 
To address label shift one can use transformation inspired by Bayes classifiers to 
adjust predictions of probabilistic classification models. 
However, there is an ambiguity how to apply this transformation in the case of  probabilistic models built 
in one-vs-one (also called pairwise) fashion. One-vs-one classification relies on the choice of a coupling
method. Some coupling methods -- called Bayes covariant --  are not affected by the ambiguity. 
Adopting a Bayes covariant method would be the ideal solution to the ambiguity. 
However, the most commonly used coupling  method of Wu-Lin-Weng is not Bayes covariant. 
In our study we present elements of theory of coupling methods and exhibit new Bayes covariant coupling methods.
One class is formed by arboreal methods. 
Additional methods can be formed by the way of ensembling. We compare their performance with that of Wu-Lin-Weng's coupling method. In majority of cases, it is possible to find a Bayes covariant method matching performance of Wu-Lin-Weng's method. }

%%================================%%
%% Sample for structured abstract %%
%%================================%%

% \abstract{\textbf{Purpose:} The abstract serves both as a general introduction to the topic and as a brief, non-technical summary of the main results and their implications. The abstract must not include subheadings (unless expressly permitted in the journal's Instructions to Authors), equations or citations. As a guide the abstract should not exceed 200 words. Most journals do not set a hard limit however authors are advised to check the author instructions for the journal they are submitting to.
% 
% \textbf{Methods:} The abstract serves both as a general introduction to the topic and as a brief, non-technical summary of the main results and their implications. The abstract must not include subheadings (unless expressly permitted in the journal's Instructions to Authors), equations or citations. As a guide the abstract should not exceed 200 words. Most journals do not set a hard limit however authors are advised to check the author instructions for the journal they are submitting to.
% 
% \textbf{Results:} The abstract serves both as a general introduction to the topic and as a brief, non-technical summary of the main results and their implications. The abstract must not include subheadings (unless expressly permitted in the journal's Instructions to Authors), equations or citations. As a guide the abstract should not exceed 200 words. Most journals do not set a hard limit however authors are advised to check the author instructions for the journal they are submitting to.
% 
% \textbf{Conclusion:} The abstract serves both as a general introduction to the topic and as a brief, non-technical summary of the main results and their implications. The abstract must not include subheadings (unless expressly permitted in the journal's Instructions to Authors), equations or citations. As a guide the abstract should not exceed 200 words. Most journals do not set a hard limit however authors are advised to check the author instructions for the journal they are submitting to.}

\keywords{ distribution shift, label shift, pairwise coupling, one-vs-one classification, Bayes classifier}

%%\pacs[JEL Classification]{D8, H51}

%%\pacs[MSC Classification]{35A01, 65L10, 65L12, 65L20, 65L70}


\ifthenelse{\not\equal{\mysecret}{2}}{
\maketitle
}{
% nothing 
}

\ifthenelse{\not\equal{\mysecret}{1}}{
\section{Introduction}

%\blindmathpaper


Distribution shift is an important problem facing machine-learning practitioners  [\cite{zhang2023dive, sugiyama2007covariate}].  Using probabilistic formulation, we say the distribution shift occurs if there is a change  in the joint probability distribution $p(X,Y)$ of covariates $X$ with the dependent variable $Y$  between training and testing. Noting the factorizations 
$$
p(X,Y) = p(Y\mid X)p(X) = p(X \mid Y )p(Y)
$$
one may distinguish the following subcategories of distribution shift:
\begin{itemize}
	\item covariate shift, if $p(X)$ changes, but $p(Y\mid X)$ is unchanged,
	\item label shift, if $p(Y)$ changes, but $P(X \mid Y)$ remains the same,
	\item concept shift, if $p(Y \mid X)$ itself changes.
\end{itemize}

We will be concerned with label shift in classification problems, where $p(Y)$ is just the prior on classes.   To provide a concrete example, consider a driver assistance system recognizing traffic signs. Suppose a speed limit sign was maliciously altered to indicate 80mph limit instead of 30mph. To avoid being fooled by the alteration, the system could utilize priors. The distribution of traffic signs will change markedly when the car leaves highway and enters streets in a city. It is very unlikely to encounter 80mph limit in a city setting. By taking into account the city priors, the system is likely to make the proper classification. In general a well-designed classification system will be able to improve its performance when supplied with a new set of class priors that correspond to the true distribution of classes. 

Change of priors can be effectively dealt with in probabilistic classifiers. The basic idea is that Bayes theorem stipulates \emph{exactly} what should happen  to the prediction of the  Bayes classifier when priors change (see \eqref{eq:changePrior}).  In practice we assume that a trained classifier closely approximates the Bayes classifier. Therefore we can make adjustment to our prediction by applying the same transformation of posterior as would happen with the Bayes classifier. 

There is a subtle issue with this approach  when handling probabilistic classifiers arising in one-vs-one classification framework.   One-vs-one classification is a two-step framework to solve  multi-class classification problems. In the first step it reduces the multiclass problem to a series of binary classification problems. In the second step it deduces the multi-class decision by aggregating decisions for the binary problems. Since the procedure is two-step, there is a possibility to apply the adjustment of priors either to the probabilistic distributions obtained after the first step, or after the second step. It is not clear what should be the preferred course of action and this problem motivates our paper. 

Some aspects of label shift were studied in earlier works including test time adaptation
\cite{vsipka2022hitchhiker}, shift detection and quantification \cite{lipton2018detecting}, statistical properties of label shift estimators \cite{garg2020unified}, and Fisher consistency of estimators of priors \cite{tasche2017fisher}.


Our approach to the problem is to evaluate practical performance of different methods to aggregate binary predictions. Such methods -- called coupling methods -- first appeared in work of \cite{refregier1991probabilistic}. Many authors followed with alternative proposals \cite{price1994pairwise, hastie1998classification,  zahorian1999partitioned, wu2004probability, vsuch2015new, vsuch2016bayes}.

We formulate the underlying principles of one-vs-one probabilistic classification in Section 2. In Section 3 we present  different classes of coupling methods. We pay special focus to methods that commute with change of priors -- Bayes covariant methods, since such methods provide an unexpected, and elegant solution to our problem. We introduce new coupling methods by forming weighted ensembles, which we prove are also Bayes covariant when a normalizing condition on weights hold (Proposition \ref{prop:ensemble}).  In Section 4 we summarize our methodology. The results of our experiments are described in Section 5. In Section 6 we summarize our findings, provide practical guidance and outline future research directions.

\section{Theoretical basis of coupling methods} \label{sec:theory}

Suppose we have $K$ classes of objects, each distributed with density $d_i$, $i=1,\ldots, K$ on a feature 
space $X$. In a multi-class classification problem we are provided with 
the prior probabilities $\Pi = (P_1, \ldots, P_K)$ of each class. Given $x$ in $X$ we aim to find 
the probability that $x$ belongs to the $i$-th class. 


\subsection{One-vs-one classification} \label{sec:one-vs-one}

An exact solution to a classification problem is provided by the Bayes classifier. To simplify the discussion we will make the following assumption.

\begin{assumption} \label{ass:1}
For all $x$ in the feature space we have $d_i(x) P_i > 0$.
\end{assumption}


The Bayes classifier predicts that the probability of $x$ belonging to the $i$-th class is
\begin{align}
 p^\textrm{multi}_i(x) = \frac{d_i(x) P_i}{\sum_{k=1}^K d_k(x)P_k}.
\end{align}

Besides the multi-class classifier we may also consider the binary Bayes classifiers for any pair $(i,j)$ of 
classes. Assuming the prior on the two classes is proportional to $(P_i, P_j)$, the binary Bayes classifier 
predicts the probability of the $i$-th class as 

\begin{align}
	p_{ij}^\textrm{binary}(x) = \frac{d_i(x) P_i}{d_i(x)P_i + d_j(x)P_j}.
\end{align}

Knowledge of vector $\boldsymbol{p}^\textrm{multi}(x)$ is equivalent to knowing the functions $p^\textrm{binary}_{ij}(x)$. Indeed under Assumption \ref{ass:1} we have
\begin{align}
p_{ij}^\textrm{binary}(x) = \frac{p_i^\textrm{multi}(x)}{p_i^\textrm{multi}(x) + p_j^\textrm{multi}(x)}. \label{eq:bt1}
\end{align}

Conversely, it is easy to show that  given values of $p_{ij}(x)$ the system of equations \eqref{eq:bt1} has a unique solution. This result (Proposition \ref{prop:binary2multi}) can be viewed as the fundamental theorem of one-vs-one classification.

\begin{prop}
	 \label{prop:binary2multi}
	 Knowledge of binary Bayes classifiers $p^\textrm{binary}_{ij}$ determines the multiclass Bayes classifier 
	 $\boldsymbol{p}^\textrm{multi}$.
\end{prop}
\begin{proof}
	Given a multiclass distribution $\boldsymbol{p} = (p_1, \ldots,p_K)$ on $K$ classes, let $\alpha_{ij}(\boldsymbol{p})$ denote the odds of $C_i$ to $C_j$:
	\begin{align}
		\alpha_{ij}(\boldsymbol{p}) = \frac{p_i}{p_j} \label{eq:alpha}
	\end{align}
	
	From \eqref{eq:bt1} we can deduce the odds of classes $C_j$ to $C_i$ for the Bayes classifier since
	\begin{align}
	\alpha_{ji}(\boldsymbol{p}^\textrm{multi}) = \frac{p_j^\textrm{multi}(x)}{p_i^\textrm{multi}(x)}
	=  \frac{d_j(x) P_j}{d_i(x) P_i} = \frac{1}{p^\textrm{binary}_{ij}(x)}- 1.
	\end{align}
	Knowing the odds of $C_2$ to $C_1$, $C_3$ to $C_2$ up to $C_{K}$ to $C_{K-1}$ together with the requirement that probabilities sum to one uniquely determines the whole vector $\boldsymbol{p}^\Pi(x)$.
\end{proof}
%\begin{proof}
%Indeed, we have 
%\begin{align*}
%\frac{1}{r_{ij}(x)} - 1 = \frac{p_j^\Pi}{p_i^\Pi}
%\end{align*}
%\end{proof}


\subsection{Coupling methods}

In practice one does not  know true distributions of samples and hence cannot 
construct Bayes classifiers. However, given a sample from the distribution, 
we can train binary classifiers to provide approximations $r_{ij}(x)$ to 
pairwise Bayes classifiers $p_{ij}^\textrm{multi}(x)$. The goal of probabilistic
one-vs-one classification is to deduce for any $x$ in the feature space $X$ an 
estimate $\hat{\boldsymbol{p}}(x)$ for $\boldsymbol{p}^\textrm{multi}(x)$. 
One does it by finding an approximate solution of the system of equations
%\begin{align}
%
%\end{align}
%
%
%Suppose that $K$ classes $C_1, \ldots, C_K$ are distributed according to probability distributions $p_i$ on a  space $X$. Let us denote the $r_{ij}(x)$ the output of Bayes (binary) classifier for $x$ in $X$ yielding the probability $p(x \in C_i \mid x\in C_i \cup C_j )$. Then we have
%\begin{equation}
%r_{ij}(x)= \frac {p_i(x)}{p_i(x) + p_j(x)}.
%\end{equation}
%
%
%The system of equations is called Bradley--Terry equations.  The equations in \eqref{eq:bt1}  can be transformed to a system of linear equations. The theoretical basis for one-vs-one classification framework is provided by the following result.
%
%
%
%Suppose we have estimators $\hat r_{ij}$ for Bayes predictions $r_{ij}$ for all pairs $i\not= j$. A \emph{coupling method} obtains a multi-class probabilistic estimate $\boldsymbol{\hat p}= (\hat p_1, \ldots, \hat p_K)$ that satisfy
\begin{equation}
	{r}_{ij}(x) \approx \frac {\hat p_i(x)}{\hat p_i(x) + \hat p_j(x)}. \label{eq:bt2}
\end{equation}

% One should note that 
%\begin{itemize}
%\item the resulting system of equations will usually not be consistent, because $\hat{r}_{ij}$ are only estimates of true values $r_{ij}$,
%\item when probabilities are parametrized in other ways (e.g. other common parametrizations are as odds, or log-odds), the system of equations \eqref{eq:bt} is non-linear.
%\end{itemize}

Equations forming system \eqref{eq:bt2} are called Bradley-Terry equations. In parallel to Assumption \ref{ass:1} it is customary to make the following assumption on estimates $r_{ij}$.

\begin{assumption} \label{ass:2}
For any $x$ in the feature space $r_{ij}(x) > 0$.
\end{assumption}

Commonly used classifiers such as linear discriminant analysis, (penalized) logistic regression or support vector machines all satisfy this assumption.
	
A method to solve Bradley-Terry equations is called a \emph{coupling method}. Its input can be represented by a matrix $\boldsymbol{R}(x)$ with off-diagonal entries being $r_{ij}(x)$. Schematically, classification using coupling method $\boldsymbol{v}$ proceeds as indicated in Figure \ref{fig:coupling}.

\begin{figure}[!h]
	\centering
	\begin{tikzcd}[column sep=3cm]
		\textrm{sample $x$}\arrow[r, "binary~classifiers"] \arrow[dr,dotted,swap, "multi-class~ prediction"] & \boldsymbol{R}(x) %\arrow{d}{coupling}[swap]{\boldsymbol{v}} \\
		\arrow{d}{coupling} \\
		 & \boldsymbol{v}(\boldsymbol{R}(x)) 
	\end{tikzcd}
	\caption{Multi-class classification using a coupling method $\boldsymbol{v}$}
	\label{fig:coupling}
\end{figure}


The system of Bradley-Terry equations is usually inconsistent and additional assumptions are needed to construct an estimator. In Sections \ref{sec:des:exact} and \ref{sec:des:inexact} we review various approaches to construct coupling methods.


\subsection{Exact desiderata on coupling method} \label{sec:des:exact}

%Let us outline what kind of exact conditions may be desirable in a coupling method. 
In %Section \ref{sec:des:exact} 
this section
 we will describe exact mathematical criteria that one may impose on a coupling method, while in section \ref{sec:des:inexact} we shall describe more flexible properties. 

\begin{itemize}
\item  unique solution(under Assumption \ref{ass:2}),
\item Bradley-Terry consistency,
\item canonicity,
\item symmetry,
\item Bayes covariance.
\end{itemize}


%\begin{itemize}
%\item a coupling method should should always provide a unique solution, at least when all $r_{ij}>0$,
%\item Bradley-Terry consistency,
%\item canonicity,
%\item symmetry,
%\item Bayes covariance,
%
%\item good accuracy on benchmark tasks,
%\item lack of parameters, or at least having a small number of parameters,
%\item Hinton's minimality,
%\item simplicity of computation.
%\end{itemize}



\emph{Unique solution} refers to the fact that the procedure to compute a multi-class probabilistic prediction always converges to a unique solution. One way to satisfy this condition would be that it amounts to a linear system of equations in $K$ variables, and the matrix of the system would be invertible. Another example would be when the procedure amounts to finding the minimum of a strongly convex function. 

\emph{Bradley-Terry consistency} refers to the outcome of a procedure providing a unique solution. The requirement states that should we apply the coupling procedure to binary Bayes classifiers, then the result is the output of multi-class Bayes classifier i.e. satisfies \eqref{eq:bt1}.

\emph{Canonicity} refers to requirement that the method should be natural. Borrowing an example from regression analysis, Gauss-Markov theorem states that ordinary least squares (OLS) estimates is best linear unbiased estimate of the coefficients of a linear model. Therefore OLS  is a \emph{canonical} way to estimate parameters of a regression model. 

Note that potentially there may be multiple canonical methods, derived from (or satisfying) differing sets of assumptions. For instance, in the theory of scoring rules, both logarithmic and quadratic scores can be viewed as canonical \cite{shannon1948mathematical,selten1998axiomatic}.

\emph{Symmetry} We can define two natural actions on permutation group on $K$-elements. Suppose $\sigma$ is a permutation on $K$ elements. 

\begin{itemize}
	\item Given  distributions on the classes $\boldsymbol{p}= (p_1, \ldots, p_K)$, we can define
	\begin{align}
		\sigma(\boldsymbol{p})= (p_{\sigma(1)},p_{\sigma(2)}, \ldots, p_{\sigma(K)}).
	\end{align}
	 \item $\sigma$ also acts on the matrix of binary predictions. Namely, the entry of $\sigma(\boldsymbol{R})$ in $i$-th row and $j$-th column is defined to be $r_{\sigma(i), \sigma(j)}$.
	 \end{itemize}
We say that a coupling method $V$ is \emph{symmetric} if 
\begin{align}
		\sigma(\boldsymbol{v}(\boldsymbol{R})) = \boldsymbol{v}(\sigma(\boldsymbol{R}))\quad\textrm{for any permutation $\sigma$}.
\end{align}

\emph{Bayes covariance} is a notion introduced in  \cite{vsuch2016bayes}. It refers to the behavior of a probabilistic multi-class method built in one-vs-one fashion, when the priors change. Any Bayes classifier, whether binary or multiclass changes its prediction when the priors change. Let us describe the process. Suppose we change priors from $\Pi = (P_1, \ldots, P_k)$ to $\Pi'= (P'_1, \ldots, P'_K)$. Then the prediction of Bayes classifiers changes from $\boldsymbol{p}^\Pi= (p_1, \ldots, p_K)$ to the vector 
\begin{align}
\boldsymbol{p}^{\Pi'} \propto \biggl(p_1 \frac{P'_1}{P_1}, \ldots, p_K \frac{P'_K}{P_K}\biggr). \label{eq:changePrior}
\end{align}

When we use a coupling method and the prior changes, we  can apply \eqref{eq:changePrior} either before or after coupling. If the results are identical, for all initial pairwise data $\boldsymbol{R}$, then we say that the coupling method is \emph{Bayes covariant}. 

Let us make this explicit. Let $\boldsymbol{v}$ be a coupling method. Denote by  $\pi$ the change of priors $\pi:\Pi \rightarrow \Pi'$. If we have a  binary classifier predicting distribution $D \propto (p, p')$ on two classes $C_i$ and $C_j$, then we define new distribution $\pi_2^{ij}(D)$ by requiring
$$
\pi_2^{ij}(D) \propto \biggl(p \frac{P'_i}{P_i},p' \frac{P'_j}{P_j}\biggr).
$$
We denote by $\boldsymbol{R}^\pi$ the $K\times K$ matrix with entries $\pi_2^{ij}(r_{ij})$. The value $\boldsymbol{v}(\boldsymbol{R}^\pi)$ represents the effect of change of priors if we apply \eqref{eq:changePrior} \emph{before} coupling.

On the other hand we can apply \eqref{eq:changePrior} after coupling. Say we have a distribution on $K$ classes
\begin{align}
D \propto (p_1, \ldots, p_K)
\end{align}
then we define $D^\pi$ by requiring 
\begin{align}
D^\pi \propto \biggl(p_1 \frac{P'_1}{P_1}, \ldots, p_K \frac{P'_K}{P_K}\biggr). \label{def:prior:effect}
\end{align}
We now say that a method is Bayes covariant if
\begin{align}
\boldsymbol{v}(\boldsymbol{R}^\pi)=	\bigl(\boldsymbol{v}(\boldsymbol{R})\bigr)^\pi,
\end{align}
for any matrix $\boldsymbol{R}$ with $r_{ij}> 0$.




%Let $\boldsymbol{R}^\pi$ be the matrix of pairwise predictions when we apply \eqref{eq:changePrior} to each binary classifier.
%For Bayes covariant method $\boldsymbol{v}$ the following diagram commutes.
%
%\begin{figure}[!h]
%\centering
%\begin{tikzcd}
%	\boldsymbol{R} \arrow[r, "\pi"] \arrow[d, "coupling"] & \boldsymbol{R}^\pi \arrow[d, "coupling"] \\
%	\boldsymbol{v}(\boldsymbol{R}) \arrow[r, "\pi"] & \boldsymbol{v}(\boldsymbol{R}^\pi) =\boldsymbol{v}(\boldsymbol{R})^\pi
%\end{tikzcd}
%\caption{Diagrammatical description of Bayes covariance for a coupling method $\boldsymbol{v}$}
%\label{fig:bc}
%\end{figure}


\subsection{Inexact desiderata for coupling methods}

\label{sec:des:inexact}

In this section we describe several concepts that lack strict mathematical delineation, but may still be important to consider.

\emph{Robustness -- good accuracy on benchmark tasks} is  a strong argument to prefer a particular coupling method. This criterion is of course subjective, because there is a great variety of datasets in practice and inevitably there will be instances where any coupling method would underperform. However a good coupling method should perform robustly across a variety of datasets. A historical  example to follow is linear discriminant (LDA) of  Sir Ronald Fisher. At the time he was submitting his paper describing LDA, he had confirmed usefulness of his method not only on the famous iris dataset, but also on two separate collaborations with archaeologists \cite{barnard1935secular, martin1936study}. 

\emph{Lack of parameters} is strongly beneficial for situations when the dataset is small and it would be imprudent to reserve a sizeable portion to obtain unbiased estimates of the parameters of the coupling method. This issue gains expediency when the underlying binary classification method requires crossvalidation to select some hyperparameters. For example, this is the case of support vector machines, where often one opts for RBF kernel, which requires choosing two hyper-parameters. 

\emph{Simplicity of computation}  may refer to the computational requirement for inference. With the dramatic increase of edge computing devices, such as smartphones, a simple, fast, energy-efficient algorithm may be preferable to a better performing one which would require more computational resources. 

Let us finally mention \emph{Hinton's minimality}. Geoffrey Hinton is reported to object to the underlying principle of one-vs-one classification, where any single classifier influences the multi-class prediction \cite[p.~467]{hastie1998classification}. Suppose we are classifying objects from CIFAR-10 dataset.  If the true class is  a ``dog'', why should we care what is the output of the classifier distinguishing ``airplanes'' from ``ships''? The binary classifier is trained only on airplanes and ships, and has never seen a dog. One may thus prefer a classifier which somehow suppresses noisy output of (seemingly) irrelevant classifiers.


%The first that the five conditions have a clear mathematical definition, whereas the rest are not exact. For instance while the ideal situation for a coupling method is to have no parameters, the presence of one or two trainable parameters may be justified, if it yields noticeably better performance.





\section{Examples of coupling methods} \label{sec:coupling}

There is a plethora of methods available for coupling. In this section we describe four major classes.

\subsection{Methods aiming to minimize binary divergence}

A multi-class estimate $\hat{\boldsymbol{p}}$ obtained by a coupling method implies via \eqref{eq:bt1} probability distributions on each pair of classes:
\begin{align*}
	\hat D_{ij}^\textrm{multi}= \biggl(\frac{\hat{p_i}}{\hat p_i + \hat p_j},\frac{\hat{p_j}}{\hat p_i + \hat p_j}\biggr)	
\end{align*}
On the other hand, binary classifiers provide another distribution for each pair of  classes, namely
\begin{align*}
\hat D_{ij}^\textrm{binary} = (\hat r_{ij}, \hat r_{ji}).
\end{align*}

A natural way to find  $\hat{\boldsymbol{p}}$ is thus to minimize a functional incorporating divergence measures between these distributions
\begin{align*}
\hat{\boldsymbol{p}} \stackrel{def}{=} \argmin_{\boldsymbol{p}} \sum_{ij} w_{ij} \divm (\hat D_{ij}^\textrm{multi}, \hat D_{ij}^\textrm{binary})
\end{align*}

A notable example which uses this principle is the coupling method introduced in [\cite{hastie1998classification}]. Their method uses Kullback-Leibler divergence $\divm_\textrm{KL}$
\begin{align*}
	\divm\nolimits_\textrm{KL} (\boldsymbol{p}, \boldsymbol{q})= \sum p_i \log (p_i/ q_i)
\end{align*}

Kullback-Leibler divergence strongly penalizes  mispredictions when the true class is predicted to have very low probability. In the context of classification, when we are primarily concerned with accuracy , one could advantageously use other Bregman divergence derived from a proper score [\cite{gneiting2007strictly, buja2005loss}].
 
For instance, a popular alternative is divergence arising from the quadratic  score [\cite{gneiting2007strictly}] (also known as Brier score [\cite{brier1950verification}]). This divergence takes the form
$$
\divm\nolimits_\textrm{quad} (\boldsymbol{p}, \boldsymbol{q})= \sum (p_i - q_i)^2.
$$
Unlike Kullback-Leibler divergence, the penalty for incorrect prediction with arbitrarily low probability is bounded (by one). This quadratic score is the basis of the popular Wu-Lin-Weng's method [\cite{wu2004probability}]. 

Their method uses an additional trick, which we term \emph{self-referentiality}. In the method of Hastie and Tibshirani, $w_{ij}$ are constants derived from observed class frequencies. However, we could use weights $w_{ij}$ that increase when the predicted probability of either class $i$ or $j$ is high. A reasonable choice would be for instance $w_{ij}= p_i + p_j$. In the case of quadratic score, it is more convenient to use weights $w_{ij} = (p_i + p_j)^2$ because then the resulting optimization problem is just minimization of a quadratic form
\begin{align*}
\hat{\boldsymbol{p}} \stackrel{def}{=} \argmin_{\boldsymbol{p}} \sum_{i,j} (r_{ij}p_j - r_{ji}p_i)^2.
\end{align*}

We note that the method is symmetric. The choice of quadratic scoring rule can also be viewed as canonical in view of axiomatic characterization given in  [\cite{selten1998axiomatic}].  However the choice of weights $w_{ij}$ to make the method computationally amenable is somewhat arbitrary. Thus we do not view Wu-Lin-Weng's method as canonical.

\subsection{Arboreal methods}

System of equations \eqref{eq:bt2} is overdetermined because there are $\binom{K}{2}$ equations together with the requirement that the total sum of predicted probabilities is one.  Given an overdetermined system of equations, a common approach is to select a minimal subset of equations needed to solve for the unknowns. This principle underlies arboreal coupling methods.

An arboreal coupling method ignores all but $K-1$ of the pairwise predictions. A simple example for 3 classes is the classifier which ignores prediction of the binary classifier distinguishing between classes 1 and 3. The two equations form \eqref{eq:bt2} together with the requirement that the prediction probabilities should sum to 1 yield the following system of equations
\begin{equation}
	\begin{split}
		\frac{\hat p_1}{\hat p_1 + \hat p_2} &= {r}_{12}\\
		\frac{\hat p_2}{\hat p_2 + \hat p_3} &= {r}_{23}\\
		\hat p_1 + \hat p_2 + \hat p_3 &= 1
	\end{split}
	\label{eq:arb1}
\end{equation}

For $K=3$, there are three different arboreal coupling methods, which we shall call $\boldsymbol{s}_1$, $\boldsymbol{s}_2$, $\boldsymbol{s}_3$, where the system \eqref{eq:arb1} represents the coupling method $\boldsymbol{s}_2$.

For $K>3$ the three methods can be generalized to corresponds to the trees having the graph structure  of a star centered at class $C_i$ for $i=1, \ldots, K$.

More generally, to obtain a regular system of equations for  any of $K$ , it is necessary that the tree corresponding to an arboreal method forms a spanning tree. Note that any arboreal method is non-canonical, because there is no natural way to choose a tree in a complete graph. In fact, a stronger statement is true -- no arboreal method is symmetric.
%However, we will see in Section \ref{sec:bc2} that by ``averaging'' several stars we can obtain a canonical 



\subsection{Methods emphasizing Hinton's minimality}

A variant on arboreal methods is the following oracle method which we shall call Hinton's oracle. It  assumes it has the access to the ground truth $y = C_t$. It outputs probabilities by solving only the equations
$$
\frac{\hat p_t}{\hat p_t + \hat p_j} = {r}_{tj},\quad \textrm{for }j\not = t,
$$
together the  with total sum equation
$$
\sum_{i=1}^K  \hat p_i = 1.
$$

For any one sample belonging to class $C_t$, its result is identical to the arboreal method where the underlying tree forms a star with center at $C_t$. This method is clearly the optimal one with respect to Hinton's minimality criterion, because it ignores the output of any classifier not trained on the true class. It is not a true classification method, however, because it ``peeks'' at the label. Its main utility is to provide a way to benchmark other methods with respect to Hinton's minimality.

In order to convert Hinton's oracle into a true classification method we can use the principle of self-referentiality.  Denote by $\boldsymbol{s}_i$ the prediction of the arboreal method, where the underlying tree is a star centred at the class $C_i$. The predictions will generally span the vector space $R^K$. In that case the prediction $\hat{\boldsymbol{p}}$ of the method  can be expressed as a linear combination of $\boldsymbol{s}_i(x)$. Let us now define \emph{radial} coupling method. The  method requires that for the  weights we could in fact use the components of $\boldsymbol{p}$ i.e.
\begin{equation}
	\begin{split}
	\hat {\boldsymbol{p}} &= \hat p_1 \boldsymbol{s}_1(x) + \ldots + \hat p_K \boldsymbol{s}_K(x)\\
	1 &= \hat p_1 + \ldots + \hat p_K
	\end{split}
	 \label{eq:radial}
\end{equation}
%
The intuition behind \eqref{eq:radial} is that if some component of $\hat{\boldsymbol{p}}$ is close to 1 (and thus the remaining components are close to zero), then the prediction $\hat{\boldsymbol{p}}$ should be close to the prediction of Hinton's oracle which is provided by ${\boldsymbol{s}}_i$.

\subsection{Bayes covariant methods} \label{sec:bc2}

There are many Bayes covariant methods. The simplest example are arboreal coupling methods. They are not canonical, but with the help of ensembling, one can construct also canonical coupling methods. 

A very helpful aid to study Bayes covariant methods is property $BC_{ij}$ (the notation stands for \emph{Bayes covariant when restricted to classes} $i$ and $j$). A coupling method $\boldsymbol{v}$ satisfies $BC_{ij}$ if for any change of priors $\pi$ and any  pairwise data $\boldsymbol{R}$ the odds of classes $C_i$ to $C_j$ predicted by $\boldsymbol{v}(\boldsymbol{R}^\pi)$ and $\bigl(\boldsymbol{v}(\boldsymbol{R})\bigr)^\pi$ are equal. Using earlier notation for odds (see \eqref{eq:alpha}), one has
\begin{align}
	\alpha_{ij} \bigl(\boldsymbol{v}(\boldsymbol{R}^\pi) \bigr) = \alpha_{ij} \bigl(\boldsymbol{v}(\boldsymbol{R})^\pi \bigr)
\end{align}

Let us list some easy to verify properties of this concept:

\begin{itemize}
	\item[a)] a Bayes covariant coupling method satisfies $BC_{ij}$ for any $i\not= j$ (see Proposition \ref{prop:bcprop} in Appendix \ref{app:bc1}),
	%\item[b)] if a coupling method satisfies $BC_{12}, BC_{23}, \ldots, BC_{(K-1),K}$ then it is Bayes covariant (see Proposition \ref{prop:bcprop}),
	\item[b)] if a coupling method satisfies $BC_{ij}$ for all $i\not= j$ then it is Bayes covariant (see Proposition \ref{prop:bcprop} in Appendix \ref{app:bc1}),
	\item[c)] if a coupling method satisfies $BC_{ij}$ and $BC_{jk}$ then it satisfies $BC_{ik}$ (see Lemma \ref{lem:transitivity} in Appendix \ref{app:bc1}).
\end{itemize}

Now we are ready to prove the our first result.

\begin{prop}
Any arboreal method is Bayes covariant.
\end{prop}
\begin{proof}

An arboreal coupling method $\boldsymbol{t}$ induced by a spanning tree $T$ on the classes satisfies $BC_{ij}$ whenever the edge $C_iC_j$ belongs to $T$. Since any pair of vertices is connected by a path, it follows from transitivity of $BC_{ij}$ (property c)\,) that the assumption of b) holds and thus the coupling method is Bayes covariant.
	
\end{proof}

Given several Bayes covariant methods one can form  their ``linear combination'' - an ensemble which is also Bayes covariant. Let us make the construction explicit.

Suppose first that real numbers $a_1, \ldots,a_M$ are real numbers and $\boldsymbol{v}_1, \ldots, \boldsymbol{v}_M$ are probabibilistic multi-class classification methods satisfying Assumption \ref{ass:2}. We define their linear combination $\bigoplus_i a_m \boldsymbol{v}_m$ as the classification method that for any $x$ yields the probability distribution on $K$ classes that satisfies
\begin{align*}
 	\bigl(a_1 \boldsymbol{v}_1 \oplus \ldots \oplus a_M \boldsymbol{v}_M\bigr)(x) \propto \boldsymbol{v}_1^{a_1}(x) \odot \cdots \odot \boldsymbol{v}_M^{a_M}(x),
\end{align*}
where the symbol $\odot$ denotes componentwise multiplication, and exponentials are computed component-wise. With this definition we can state the following result.

\begin{prop}
Suppose the coupling methods $\boldsymbol{v}_1, \ldots, \boldsymbol{v}_M$ are Bayes covariant. If $\sum a_m = 1$, then $\boldsymbol{e} = \bigoplus_m a_m \boldsymbol{v}_m$ is also a Bayes covariant coupling method.
\end{prop}

\begin{proof}
It is sufficient to show that $BC_{ij}$ holds for any $i\not= j$. 

Let $\pi$ be a change of priors from $(P_1, \ldots, P_K)$  to $(P'_1, \ldots, P'_K)$. Let $o_m$ denotes the odds of classes $C_i$ and $C_j$ predicted by coupling method $\boldsymbol{v}_m$ i.e.
$$
o_m = \alpha_{ij}(\boldsymbol{v}_m(\boldsymbol{R})).
$$

It follows from the definition of linear combination of coupling methods that the odds of $C_i$ to $C_j$ for $\boldsymbol{e}$ are
\begin{align}
\prod_m o_m^{a_m}.
\end{align}
It follows that the odds of $\boldsymbol{e}^\pi$ are 
\begin{align}
\frac{P'_i}{P_i} \frac{P_j}{P'_j} \prod_m o_m^{a_m}.
\end{align}

Let us now use Bayes covariance of $\boldsymbol{v}_i$. 
We have
\begin{align}
	\biggl(\bigoplus_m a_m \boldsymbol{v}_m\biggr)(\boldsymbol{R^\pi}) & \propto 
	\boldsymbol{v}_1^{a_1}(\boldsymbol{R}^\pi) \odot \cdots \odot \boldsymbol{v}_M^{a_M}(\boldsymbol{R}^\pi) =  \bigodot_m  (\boldsymbol{v}_m^\pi(\boldsymbol{R}))^{a_m} \\
%	&= \frac{P'_i}{P_i} \frac{P_j}{P'_j} \prod_m o_m^{a_m}
\end{align}

It follows that the odds of $C_i$ to $C_j$ for $\bigoplus_m a_m \boldsymbol{v}_m$ are 
\begin{align}
\biggl(\frac{P'_i}{P_i} \frac{P_j}{P'_j} \biggr)^{\sum a_m} \prod_m o_m^{a_m}.
\end{align}
Clearly, under the assumption $\sum_m a_m = 1$, the classifier $\bigoplus_m a_m \boldsymbol{v}_m$ satisfies $BC_{ij}$. Since this holds for any $i\not=j$,  it follows that $\boldsymbol{e}$ is Bayes covariant.
\end{proof}


% 
% given such $\alpha + \beta = 1$. If the prediction of the two Bayes covariant methods $\boldsymbol{B}$ and $\boldsymbol{\tilde B}$ are
%\begin{align*}
%	\boldsymbol{B}(x) = (p_1, \ldots, p_K)\quad\textrm{and}\quad \boldsymbol{\tilde B}(x) = (\tilde p_1, \ldots, \tilde p_K)
%\end{align*}
%then the prediction of the ensemble $\alpha 	\boldsymbol{B} \oplus \beta \boldsymbol{\tilde B}$ is given by
%\begin{align*}
%	\bigl(\alpha 	\boldsymbol{B} \oplus \beta \boldsymbol{\tilde B}\bigr)(x)\propto (\,p_1^\alpha \tilde p_1^\beta, \ldots, p_K^\alpha \tilde p_K^\beta).
%\end{align*}

As an example, consider the classification problem with three classes. We have three arboreal classifiers $\boldsymbol{s}_1, \boldsymbol{s}_2, \boldsymbol{s}_3$ which correspond to star graphs centered respectively at classes $C_1, C_2, C_3$. The classifier 
\begin{align}
\boldsymbol{e}_3 \stackrel{def}{=} \frac13 \boldsymbol{s}_1 \oplus \frac13 \boldsymbol{s}_2 \oplus \frac 13 \boldsymbol{s}_3  \label{eq:bc1}
\end{align}
is Bayes covariant \emph{and} symmetric. A key result proved in [\cite{vsuch2016bayes}] is that it is a unique  classifier having these two properties.

\begin{thm} \label{thm:K3}
	There exists a unique symmetric Bayes covariant coupling methods for $K=3$.
\end{thm}

\begin{proof}
Let 
\begin{align}
	\boldsymbol{R}_s \stackrel{def}{=} \begin{pmatrix} \cdot & s & 1 -s \\  1-s & \cdot & s \\ s & 1-s & \cdot \end{pmatrix}
\end{align}
Since $R_s$ is invariant under the action of any permutation of 3 elements, 
for a symmetric coupling method we have 
\begin{align}
	\boldsymbol{v} (\boldsymbol{R}_s) = \biggl(\frac 13, \frac 13, \frac 13 \biggr).
\end{align}
Now applying  Proposition \ref{prop:bc3} from Appendix \ref{app:bc1} yields uniqueness. The existence of the method was already shown above.
\end{proof}


Let us point out one consequence of the theorem. We could define another ensemble $\tilde{\boldsymbol{e}}$ by averaging the predictions of $\boldsymbol{s}_1, \boldsymbol{s}_2, \boldsymbol{s}_3$:
\[
\tilde{\boldsymbol{e}} = \frac13 \boldsymbol{s}_1(x) + \frac13 \boldsymbol{s}_2(x) + \frac 13 \boldsymbol{s}_3(x).
\]
Note that here we use the plus symbol to denote the vector addition of vectors representing $K$-class probability distributions. This is a different classifier from the one given by  \eqref{eq:bc1} (see Appendix \ref{app:explicit}), and since it is symmetric, it cannot be Bayes covariant.

The classifier $\boldsymbol{e}_3$ has a canonical generalization for any $K>3$, namely the classifier
\begin{align}
\boldsymbol{e}_K \stackrel{def}{=} \frac1K \boldsymbol{s}_1 \oplus \frac1K \boldsymbol{s}_2 \oplus \ldots \oplus \frac 1K \boldsymbol{s}_K  \label{eq:bc1}
\end{align}
We call this coupling method the \emph{normal} coupling method.

\subsection{Properties of coupling methods}


The following table indicates which properties are enjoyed by each coupling method we described in this section.

\begin{table}[!ht]
\begin{tabular}{cm{2.5cm}m{1.5cm}m{1.5cm}ccm{1.5cm}m{1.5cm}}
&property & Hastie-Tibshirani & Wu-Lin-Weng & radial & normal & arboreal & Hinton's oracle \\
\hline 
\multirow{5}{*}{\begin{turn}{90}\makecell{exact}\end{turn}}
&uniqueness &  yes & yes & yes & yes & yes & yes \\
&Bradley-Terry consistency & yes & yes & yes & yes & yes & yes \\
&canonicity & yes & no & yes & yes & no & yes \\
&symmetry & yes & yes & yes & yes & no & yes \\
& Bayes covariant & no & no & no & yes & yes & yes \\
\hline
\multirow{5}{*}{\begin{turn}{90}\makecell{non-exact}\end{turn}}
&Hinton's minimality & ? & ?  & ++ & ?  & ? & +++ \\
&lack of parameters & yes & yes & yes & yes & yes & yes \\
& amounts to a linear system & no & yes & yes & yes & yes & yes\\
& linear system of special form & no & no & yes & yes & yes & yes \\
& iterative method & yes & yes & yes & no & not needed & not needed\\
\hline
\end{tabular}
\caption{Summary of properties of coupling methods described in Section \ref{sec:coupling}.}
\label{tab:summaryCoupling}
\end{table}



\section{Experiments}

The choice of a coupling method is an important decision for practitioners. We aim to investigate the following interrelated questions:

\begin{itemize}
	\item[(a)] Which coupling methods without parameters show good performance across diverse data sets? Is it the orthodox method of Wu-Lin-Weng? An experiment to answer this question is carried out Section \ref{sec:exp1}. 
	\item[(b)] If the answer to a) is a coupling method which is not Bayes covariant, can we find a parametric Bayes covariant method which  matches its performance? An experiment related to this question is carried out in  Sections \ref{sec:exp2} and \ref{sec:exp3}.
	\item[(c)] If the answer to b) is negative, is it better to apply the change of priors before or after coupling? Additionally, how serious is the impact of the lack of Bayes covariance in practice. This question is investigated in Section \ref{sec:exp4}.
\end{itemize}

These questions will be put to test in Sections \ref{sec:exp1}--\ref{sec:exp4}. The methodology of our experiments is described in the next two subsections.

\subsection{Datasets}

It is natural to consider the benchmark datasets used in previous evaluation of coupling methods \cite{wu2004probability}. The authors considered altogether seven datasets with number of classes ranging from 3 to 26. For each dataset they extracted 40 pairs of training and testing datasets. Half of these were smaller (300 training samples and 500 testing samples) and the other half were larger (800 training samples and 1000 testing samples). In our experiments we use only the larger pairs. 

As a preliminary step we conducted analysis of linear separability of the datasets. We used ECOS solver in CVXR library to solve the underlying quadratic program. The results are reported in Table \ref{tab:sep}.

% latex table generated in R 4.5.1 by xtable 1.8-4 package
% Wed Aug 13 15:27:15 2025
\begin{table}[ht]
\centering
\begin{tabular}{lrr}
  \hline
Dataset & $K$ & Percentage linearly separable \\ 
  \hline
dna &    3 & 100 \%    \\ 
  letter &   26 & 99 \%    \\ 
  mnist &   10 & 100 \%    \\ 
  satimage &    6 & 84 \%    \\ 
  segment &    7 & 89 \%    \\ 
  usps &   10 & 100 \%    \\ 
  waveform &    3 & 0 \%    \\ 
   \hline
\end{tabular}
\caption{The benchmark datasets with indication of the number of classes $K$ and the percentage of binary classification problems that are linearly separable} 
\label{tab:sep}
\end{table}


We can infer that except for 'waveform' dataset, majority of binary subproblems are linearly separable.

\subsection{Classification methodology}

We have opted to use LDA as a binary classification tool. Its appeal is that it is applicable even for linearly separable datasets, thus obviating additional crossvalidation step incurred by methods that use regularization (e.g. penalized logistic regression or SVM). It usually provides results close to logistic regression \cite{james2013introduction}.

For preprocessing we first removed features that were constant across classes. Then we projected the data using PCA to the subspace for which corresponding singular values were $> tol$, where $tol$ is a hyper-parameter of an experiment. We used PCA without scaling or centering since the datasets' features were already normalized to interval $[-1,1]$. 

For training LDA we used MASS library in R. We implemented coupling methods in C++ using Eigen library. As arguments to the coupling methods we use log-odds:
$$
\tilde r_{ij} = \log \biggl(\frac{1}{r_{ji}} - 1\biggr).
$$
The resulting matrix $\tilde{\boldsymbol{R}}$ is skew symmetric (with the convention of having zeroes on the diagonal).

This parametrization allows for precise representation of extremely small probabilities which are sometimes produced by LDA. These small probabilities might be expected given that a large fraction of binary problems is linearly separable. 

Small probabilities generally have very little effect on the implementation of Bayes covariant methods which generally operate using log-odds. However small probabilities do affect both the method of Wu-Lin-Weng as well as radial method, because then Assumption \ref{ass:2} fails to hold. 

%\begin{figure}[!ht]
%	\includegraphics[width = 0.5\linewidth]{graph/dna-lda.pdf}
%	\includegraphics[width = 0.5\linewidth]{graph/waveform-lda.pdf}
%	\caption{Projections of three-class datasets using LDA Left: dna, right: waveform}
%\end{figure}


\subsection{Experiment: Accuracy of canonical coupling methods} \label{sec:exp1}

The first experiment compares the performance of the following canonical coupling methods:
\begin{itemize}
\item method of Wu-Lin-Weng, which is widely used in software libraries, and thus can be viewed as an orthodox choice,
\item normal coupling, which is Bayes covariant,
\item radial coupling, which is not Bayes covariant,
\item Hinton's oracle.
\end{itemize}

% % latex table generated in R 4.4.1 by xtable 1.8-4 package
% Sun Apr  6 13:12:57 2025
\begin{table}[ht]
\centering
\begin{tabular}{lrrr}
  \hline
dataset & normal & radial & Wu-Lin-Weng \\ 
  \hline
dna & 0.81 & 0.84 & 0.84 \\ 
  letter & 0.35 & 0.75 & 0.77 \\ 
  mnist & 0.48 & 0.47 & 0.50 \\ 
  satimage & 0.68 & 0.81 & 0.81 \\ 
  segment & 0.57 & 0.93 & 0.94 \\ 
  usps & 0.22 & 0.48 & 0.78 \\ 
  waveform & 0.86 & 0.86 & 0.86 \\ 
   \hline
\end{tabular}
\caption{Comparison of accuracy for three parameterless coupling methods} 
\label{tab:multi}
\end{table}

The results are shown in Figure \ref{fig:exp1-plot1}. We can clearly see that Wu-Lin-Weng's method has the most robust performance. On some datasets it trounces normal coupling method by a wide margin. The edge over the radial method is smaller with only mnist and usps datasets showing a significant difference for smaller values of $tol$ hyperparameter.  However in view of Theorem \ref{thm:K3} the Wu-Lin-Weng's method is not Bayes covariant, therefore we will extend the set of Bayes covariant methods beyond the canonical normal coupling.

\begin{figure}
	\includegraphics{graphs/exp1-plot1.pdf}
	\caption{Accuracy of classification plotted against the logarithm of $tol$ hyper-parameter, evaluated for different datasets.}
	\label{fig:exp1-plot1}
\end{figure}

\subsection{Experiment: Accuracy of Bayes covariant methods without parameters}
 \label{sec:exp2}


In this experiment we compare  Bayes covariant coupling methods \emph{without} parameters that could match the performance of Wu-Lin-Weng's method. We investigate  the case of three class datasets. This is a natural starting point, since it is the smallest number of classes where the question is meaningful to ask. All experiments use value of 0.05 for $tol$ hyperparameter based on results of Section \ref{sec:exp1}.  In our comparison we include the normal method, as well as the three arboreal methods $ \boldsymbol{s}_1,~ \boldsymbol{s}_2,~ \boldsymbol{s}_3$ that are not canonical. We consider all possible triples of classes in all datasets. The resulting statistics are shown in Table \ref{tab:step2}. 

% latex table generated in R 4.5.1 by xtable 1.8-4 package
% Thu Aug 14 09:38:32 2025
\begin{table}[ht]
\centering
\begin{tabular}{lr}
  \hline
Outcome & Percentage \\ 
  \hline
 $\boldsymbol{e}_3$ at least as accurate as $\boldsymbol{wlw}$ & 24 \%   \\ 
  one of $\boldsymbol{s}_1, \boldsymbol{s}_2, \boldsymbol{s}_3$ at least as accurate as $\boldsymbol{wlw}$ & 53 \%   \\ 
  $\boldsymbol{wlw}$ more accurate than $\boldsymbol{e}_3, \boldsymbol{s}_1, \boldsymbol{s}_2, \boldsymbol{s}_3$ & 39 \%   \\ 
   \hline
\end{tabular}
\caption{Comparison of Wu--Lin--Weng's method $\boldsymbol{wlw}$ and  parameterless coupling methods on three class subsets. } 
\label{tab:step2}
\end{table}


We can see that in slightly less than half of the cases none of the four parameterless Bayes covariant methods was able to match the performance of Wu-Lin-Weng's method. This leads us to extend evaluation to Bayes covariant methods with parameters. 

\subsection{Experiment: A parametric family of Bayes covariant methods}
\label{sec:exp3}

Let us define a two-parameter family of Bayes covariant coupling methods
\begin{align}
a_1 \boldsymbol{s}_1 \oplus a_2 \boldsymbol{s}_2 \oplus a_3 \boldsymbol{s}_3,\quad\textrm{where } a_1 + a_2 +a_3 = 1. \label{eq:family}
\end{align}
Note that this family includes the normal coupling method \eqref{eq:normal3} as well as the three arboreal methods $\boldsymbol{s}_1, \boldsymbol{s}_2, \boldsymbol{s}_3$.

We start by selecting 1000 three class subsets from all datasets for problems where normal method was outperformed by Wu-Lin-Weng's method. For each three-class classification problem we will try to match the performance of Wu-Lin-Weng method with a Bayes covariant method in family \eqref{eq:family}.

Selecting parameters for methods in family \eqref{eq:family} requires several methodological choices. 
\begin{itemize}
\item To avoid bias in selection of parameters $a_1, a_2, a_3$ we use a subset of testing dataset to find their optimal values. As a result, we will be comparing the accuracy of Wu-Lin-Weng method evaluated with repeated five-fold cross-validation accuracy of a method in family \eqref{eq:family}. We use 20 repetitions.
\item To compare two parameter vectors, we compare their resulting cross-validation accuracy. An alternative approach would be to use  a gradient based optimization for a smooth proxy (e.g. cross-entropy) of empirical accuracy. We opted for the former because the problem is two-dimensional, and we can use grid search instead.
\item Our choice of grid search necessitates a selection of a compact subset of parameter space that will be sampled. We choose to examine points in the simplex defined by  $a_i\geq 0$, for $i=1,2,3$. As Table \ref{tab:step2} shows, this simplex contained in more than half cases a Bayes covariant method matching the performance of Wu-Lin-Weng's method.
\end{itemize}


\begin{figure}[!ht]
\includegraphics{exp2-plot2.pdf}
\caption{Comparison of accuracy for parametric Bayes covariant methods \eqref{eq:family} vs. Wu-Lin-Weng's method}
\label{fig:par-bc}
\end{figure}

The summary statistics are shown in Figure \ref{fig:par-bc}.
We can see that only in minority of cases we found a parametric Bayes covariant coupling method matching the performance of Wu-Lin-Weng's method. 

%\subsection{Experiment: }
%
%We determined that among the 1000 three-class problems we have examined, the largest gap between Bayes-covariant parametric methods and Wu-Lin-Weng method occurred for mnist dataset and classes 5,6,9. Figure 
%
%\begin{itemize}
%	\item We searched within a much larger rectangle $D$ containing the simplex $a_i\geq 0$, which in this case is a triangle $T$.
%	\item We used Brier score as a smooth proxy for measuring the quality of multi-class classifier. This score is much smoother than 0-1 score which measures accuracy.
%\end{itemize}
%
%The results are shown in Figure \ref{fig:score}
%
%
%\begin{figure}[!ht]
%	\includegraphics{graphs/exp2-detail.pdf}
%	\caption{Color map indicates the value of Brier score computed for an ensemble of coupling methods on the testing dataset. The vertices triangle indicate the arboreal coupling methods, and the black circle the location of the optimum.}
%	\label{fig:score}
%\end{figure}
%
%We can see that indeed the optimum over the rectangle lies outside of the simplex. Let us compare the resulting accuracy of the optima in $\Delta$ and $D$. The results are shown in Table \ref{tab:step5}.
%
%\input tab-step5.tex

\subsection{Experiment: Effect of change of priors}  \label{sec:exp4}
%\subsection{Wu-Lin-Weng's method}

We have shown that are situations when Wu-Lin-Weng's method outperforms even the best Bayes-covariant parametric methods. In this section we thus ponder the question whether some non-Bayes covariant method would be more resilients to change of priors. We consider two facets to this question.

Let us start with some notation. Consider a classification method $\boldsymbol{c}_0$ based on coupling. Suppose the priors change, and let us denote by $\boldsymbol{c}_1$ the classification model where we apply \eqref{eq:changePrior} to binary classifiers, and by $\boldsymbol{c}_2$ the classification method when we apply \eqref{eq:changePrior} after coupling. 

The first question is whether for some coupling method the difference in predictions (we call it \emph{spread}) between $\boldsymbol{c}_1$ and $\boldsymbol{c}_2$ would be smaller. To gauge this question we apply change of priors by increasing tenfold the prior for every class in turn.  The averaged results are shown in Table \ref{tab:exp4}. 

% latex table generated in R 4.4.2 by xtable 1.8-4 package
% Mon Jun 16 08:56:49 2025
\begin{table}[ht]
\centering
\begin{tabular}{lrrrrrrr}
  \hline
method & dna & letter & mnist & satimage & segment & usps & waveform \\ 
  \hline
normal & 0.000 & 0.000 & 0.000 & 0.000 & 0.000 & 0.000 & 0.000 \\ 
  radial & 0.022 & 0.039 & 0.020 & 0.014 & 0.009 & 0.014 & 0.002 \\ 
  wlw2 & 0.015 & 0.044 & 0.030 & 0.013 & 0.007 & 0.020 & 0.002 \\ 
   \hline
\end{tabular}
\caption{Comparison of the difference in accuracies between methods $\boldsymbol{c}_1$ and
             $\boldsymbol{c}_2$} 
\label{tab:exp4}
\end{table}


The results, as expected, detect no change between $\boldsymbol{c}_1, \boldsymbol{c}_2$ for normal coupling method. However, there is no consistent difference in spread between radial and Wu-Lin-Weng's coupling method. Spread for the radial method is lower for 3 datasets (letter, mnist, usps datasets), higher for 3 datasets (dna, satimage, segment) and tied for waveform dataset.

The second question is whether a coupling method would more consistently provide true answers. Let us remark under Assumption 1 the Bayes classifier \emph{should} change prediction of any sample if the change of priors is sufficiently large. Thus to evaluate this question 

\begin{figure}[!ht]
	\includegraphics{exp4-truth.pdf}
	\caption{Difference in accuracies of $\boldsymbol{c}_2$ and $\boldsymbol{c}_1$ for radial and Wu-Lin-Weng's coupling method.}
	\label{fig:score}
\end{figure}


\section{Discussion}

In this paper we set out to investigate how to handle change of priors in one-vs-one probabilistic models. Let us first summarize our contributions, both theoretical and experimental.

\subsection{Theoretical insights}

In Section \ref{sec:theory} we have formulated the theoretical basis of one-vs-one classification. In Section \ref{sec:des:exact} we put down the basis for axiomatic development of coupling methods. In complementary Section \ref{sec:des:inexact} we formulated properties of coupling methods that are likely of importance in practice.

Important examples of coupling methods has been presented in Section \ref{sec:coupling}. We were able to provide a unifying formulation for method of Hastie--Tibshirani and Wu--Lin--Weng. We introduced arboreal coupling methods, which are an important example of methods without parameters, albeit noncanonical. We recast previously known normal coupling method in a new way. Although this is a canonical example of Bayes covariant coupling methods,  it is lacking in practical performance, as  we uncovered in the experimental section. This creates impetus to study more general Bayes covariant methods. The  ensembling approach we proposed in Section \ref{sec:bc2} allows one to build parametric families of Bayes covariant methods.

\subsection{Experimental findings} 

The results highlight key findings from the experimental section.

\begin{itemize}
	\item The orthodox method of Wu--Lin--Weng dominates others across diverse classification tasks.
	\item The only method close to Wu--Lin--Weng is the radial coupling method.
	\item The concern of Hinton about one-vs-one classification is unwarranted. The method that ignores binary classifiers not trained on a given sample (we call it Hinton's oracle) underperforms the method of Wu--Lin--Weng.
	\item In more than half of classification problems we were able to find a Bayes covariant coupling method which matches performance of the Wu-Lin-Weng method. 
\end{itemize}



\subsection{Practical recommendation}

Let us now turn to recommendation to practitioners on how to handle change of priors in one-vs-one probabilistic models. It is possible to choose any of the following approaches

\begin{itemize}
\item[1.] One can treat the decision on when to apply change of priors---whether before or after coupling---as a boolean hyperparameter and use standard machine learning techniques to find the optimal value.
\item[2.] One can investigate applicability of a Bayes covariant method. 
\end{itemize}
%\subsection{Properties of coupling methods}

In order to ease the second approach,   we now summarize properties of parameterless coupling methods discussed in our work.  Table \ref{tab:summaryCoupling} concisely lists the relevant properties enjoyed by each method. Note that we have split the desideratum ``simplicity of computation'' into three different aspects:
\begin{itemize}
	\item whether a coupling method amounts to a linear system of equations,
	\item if the coupling method amounts to a linear system of equations, whether the equations are of a special form (e.g., a Markov matrix) that facilitates their solution,
	\item whether the authors who published the method provide an iterative method that implements the estimate. Iterative methods are often  faster than Gaussian elimination, and less prone to numerical issues.
\end{itemize}

%interpretation of exact desiderate is unambiguous. 


%The following table indicates which properties are enjoyed by each coupling method we described in this section.

\begin{table}[!ht]
	\begin{tabular}{cm{2.5cm}|m{1.5cm}ccm{1.5cm}|m{1.5cm}}
	%	&& less competitive & \multicolumn{4}{c|}{competitive method} & oracle \\
		\hline 
		&methods % & Hastie-Tibshirani 
			& Wu--Lin--Weng & radial & normal & arboreal & Hinton's oracle \\
		\hline 
		&source % & \cite{hastie1998classification} 
			& \cite{wu2004probability} & \cite{vsuch2015new} & \cite{vsuch2016bayes} & this work & this work \\
		\multirow{5}{*}{\begin{turn}{90}\makecell{exact}\end{turn}}
		&uniqueness % &  yes 
			& yes & yes & yes & yes & yes \\
		&Bradley--Terry consistency %& yes 
			& yes & yes & yes & yes & yes \\
		&canonicity %& yes 
			& no & yes & yes & no & yes \\
		&symmetry % & yes 
			& yes & yes & yes & no & yes \\
		& Bayes covariant %& no 
			& no & no & yes & yes & yes \\
		\hline
		\multirow{5}{*}{\begin{turn}{90}\makecell{non-exact}\end{turn}}
		&Hinton's minimality %& N/A 
			& \multicolumn{5}{c}{see Section \ref{sec:exp1}} \\
		&robustness %& N/A 
			& \multicolumn{5}{c}{see Section \ref{sec:exp1}} \\
		&lack of parameters %& yes 
			& yes & yes & yes & yes & yes \\
		& amounts to a linear system % & no 
			& yes & yes & yes & yes & yes\\
		& linear system of special form % & N/A 
			& no & yes & yes & yes & yes \\
		& iterative method %& yes 
			& yes & yes & no & not needed & not needed\\
		\hline
	\end{tabular}
	\caption{Summary of properties of coupling methods described in Section \ref{sec:coupling}. The method of Hastie-Tibshirani is omitted from evaluation in Section \ref{sec:exp1}, because earlier evaluation showed it lacks robustness compared to the method of Wu-Lin-Weng \cite{wu2004probability}.}
	\label{tab:summaryCoupling}
\end{table}


\subsection{Future research}

Based on our findings we see two immediate pathways for future research. First, one can study arboreal methods in more detail, since these are Bayes covariant and as shown in Section \ref{sec:exp2} they often provide performance matching that of Wu-Lin-Weng's method. Secondly, there is a continuum of Bayes covariant methods which we barely sketched. Their properties, parameter selection,  and applicability to practical tasks need to investigated in greater detail.



%
%One-vs-one classification framework is commonly used by machine learning practitioners, for instance whenever one uses radial basis function SVM implemented in  standard libraries. Since those method rely on coupling method of Wu-Lin-Weng, one is faced with the decision how to handle differing predictions when priors change.
%
%Of course, one option is to use other classification models. For instance, random forests provide a flexible alternative. Nevertheless, RBF SVM are still one of the best choices for classification tasks with non-linear boundaries. 
%
%We have shown in Experiment 1, that substituting parameterless Bayes covariant method for Wu-Lin-Weng coupling methods incurs penalty in accuracy metric. 
%
%In our Experiment 2 we showed that the penalty can be often, but not always, eliminated when one opts for one of the parametric Bayes covariant methods.
%
%Finally, in Experiment 3 we showed that the impact of the 








}{
}


\backmatter
\bmhead{Acknowledgements}

Acknowledgements are not compulsory. Where included they should be brief. Grant or contribution numbers may be acknowledged.

Please refer to Journal-level guidance for any specific requirements.

\ifthenelse{\not\equal{\mysecret}{2}}{

\section*{Declarations}

Some journals require declarations to be submitted in a standardised format. Please check the Instructions for Authors of the journal to which you are submitting to see if you need to complete this section. If yes, your manuscript must contain the following sections under the heading `Declarations':

Ondrej Šuch and Ali Haidar were  partially supported by grants VEGA 2/0172/22 Classification using ensembles of neural networks and VEGA 2/0056/25 Cycles and edge colorings of cubic graphs. Peter Novotný did not receive support from any organization for the submitted work.

\begin{itemize}
\item Funding Ondrej Šuch and Ali Haidar were  partially supported by grants VEGA 2/0172/22 Classification using ensembles of neural networks and VEGA 2/0056/25 Cycles and edge colorings of cubic graphs. Peter Novotný did not receive support from any organization for the submitted work.
\item Conflict of interest/Competing interests Authors declare no competing interests.
\item Ethics approval and consent to participate Not applicable
\item Consent for publication Not applicable
\item Data availability All data is available from https://www.csie.ntu.edu.tw/~cjlin/papers/svmprob/data/, and also upon request from the authors.
\item Materials availability Not applicable
\item Code availability code is available upon request
\item Author contribution Not applicable
\end{itemize}

\noindent
If any of the sections are not relevant to your manuscript, please include the heading and write `Not applicable' for that section. 

%%===================================================%%
%% For presentation purpose, we have included        %%
%% \bigskip command. Please ignore this.             %%
%%===================================================%%
\bigskip
\begin{flushleft}%
Editorial Policies for:

\bigskip\noindent
Springer journals and proceedings: \url{https://www.springer.com/gp/editorial-policies}

\bigskip\noindent
Nature Portfolio journals: \url{https://www.nature.com/nature-research/editorial-policies}

\bigskip\noindent
\textit{Scientific Reports}: \url{https://www.nature.com/srep/journal-policies/editorial-policies}

\bigskip\noindent
BMC journals: \url{https://www.biomedcentral.com/getpublished/editorial-policies}
\end{flushleft}

}{
}

\ifthenelse{\not\equal{\mysecret}{1}}{

\begin{appendices}

\appendix

\section{}
\label{app:bc1}

In this appendix we provide proofs of statements used in Section \label{sec:bc2}.

\begin{prop} \label{prop:bcprop}
	Let $\boldsymbol{v}$ be a coupling method. If Assumption \ref{ass:1} holds, the following statements are equivalent:
	\begin{itemize}
		\item[a)] The coupling method $\boldsymbol{v}$ is Bayes covariant.
		\item[b)] The coupling method $\boldsymbol{v}$ satisfies $BC_{ij}$ for any pair of integers  $i\not= j$ with $1\leq i,j \leq K$.
		\item[c)] The coupling method $\boldsymbol{v}$ satisfies $BC_{12}, BC_{23}, \ldots, BC_{K-1,K}$.
	\end{itemize}
\end{prop}
\begin{proof}
It follows directly from the definition of $BC_{ij}$ that a) implies b). Also, clearly b) implies c). If the assumptions of c) hold, then the vectors $\boldsymbol{v}(\boldsymbol{R})^\pi$ and $ \boldsymbol{v}(\boldsymbol{R}^\pi)$ are proportional, but since both have components adding up to one, it follows they are equal and thus a) holds.
\end{proof}

\begin{lem} \label{lem:transitivity}
	Suppose Assumption \ref{ass:1} holds. If a coupling method $\boldsymbol{v}$ satisfies $BC_{ij}$ and $BC_{jk}$ then it satisfies $BC_{ik}$.
\end{lem}
\begin{proof}
	Let $\pi$ be a change of priors. From the assumptions we have 
	\begin{align}
		\alpha_{ij}(\boldsymbol{v}(\boldsymbol{R})^\pi ) &= \alpha_{ij}(\boldsymbol{v}(\boldsymbol{R}^\pi) ) \\
		\alpha_{jk}(\boldsymbol{v}(\boldsymbol{R})^\pi ) &= \alpha_{jk}(\boldsymbol{v}(\boldsymbol{R}^\pi) )
	\end{align}
	Since $\alpha_{ik}(\boldsymbol{p}) = \alpha_{ij}(\boldsymbol{p})  \alpha_{jk}(\boldsymbol{p})$ it follows that for any $\boldsymbol{R}$
	\begin{align}
		\alpha_{ik}(\boldsymbol{v}(\boldsymbol{R})^\pi ) &= \alpha_{ik}(\boldsymbol{v}(\boldsymbol{R}^\pi) )
	\end{align}
	which means that the coupling method satisfies $BC_{ik}$.
\end{proof}

The following proposition restricts the set of Bayes covariant coupling methods for $K=3$.

\begin{prop} \label{prop:bc3}
	For $K=3$ any Bayes covariant coupling method is uniquely determined by its prediction for  matrices of the form
	$$
	\boldsymbol{R_s} = \begin{pmatrix} \cdot & s & 1 -s \\  1-s & \cdot & s \\ s & 1-s & \cdot \end{pmatrix}.
	$$
\end{prop}
\begin{proof}
	Suppose $\boldsymbol{v}$ is a coupling method,  the priors are $\Pi = (P_1, P_2,P_3)$, and the binary classifiers yield prediction matrix 
	\begin{align}
		\boldsymbol{R} = \begin{pmatrix} \cdot & r_{12} & 1 - r_{31} \\ 1-r_{12} & \cdot & r_{23} \\
			r_{31} & 1- r_{23} & \cdot \end{pmatrix}.
	\end{align}
	
	Suppose there was a change of priors $\pi$ such that $\boldsymbol{R}^\pi$ would be belong to the one-parameter family $\boldsymbol{r}_s$ with $s$ in $(0,1)$. Then we have
	$$
	\boldsymbol{v}(\boldsymbol{R})^\pi = \boldsymbol{v}(\boldsymbol{R}^\pi) = \boldsymbol{v}(\boldsymbol{R}_s).
	$$
	Inverting action of $\pi$ would then determine $\boldsymbol{v}(\boldsymbol{R})$.
	
	In order to construct such $\pi$ let us introduce a new parametrization of the matrix of binary  predictions
	\begin{align}
		\begin{split}
			q_{12} &= \frac{1}{r_{12}} -1 \\
			q_{23} &= \frac{1}{r_{23}} -1 \\
			q_{31} &= \frac{1}{r_{31}} -1 
		\end{split}
	\end{align}
	Under the change of priors $\pi:\Pi \rightarrow \Pi' = (P'_1, P'_2, P'_3)$ we have
	\begin{align}
		q'_{ij}	= q_{ij} \cdot \frac{P_i'}{P_i} \frac{P_j}{P'_j}.
	\end{align}
	One checks easily that $\Pi' \propto (1, \frac{q_{12}}{q_{23}}, \frac{q_{12}}{q_{23}})$ yields the desired change of priors.
	
\end{proof}


%\section{}
%\label{app:explicit}
%
%% Note: in this sample, the section number is hard-coded in. Following
%% proper LaTeX conventions, it should properly be coded as a reference:
%
%%In this appendix we prove the following theorem from
%%Section~\ref{sec:textree-generalization}:
%
%In this appendix we provide explicit predictions of coupling methods used in the paper.
%
%
%
%In this appendix we prove the following theorem from
%Section~6.2:
%
%\noindent
%{\bf Theorem} {\it Let $u,v,w$ be discrete variables such that $v, w$ do
%	not co-occur with $u$ (i.e., $u\neq0\;\Rightarrow \;v=w=0$ in a given
%	dataset $\dataset$). Let $N_{v0},N_{w0}$ be the number of data points for
%	which $v=0, w=0$ respectively, and let $I_{uv},I_{uw}$ be the
%	respective empirical mutual information values based on the sample
%	$\dataset$. Then
%	\[
%	N_{v0} \;>\; N_{w0}\;\;\Rightarrow\;\;I_{uv} \;\leq\;I_{uw}
%	\]
%	with equality only if $u$ is identically 0.} \hfill\BlackBox
%
%\section{}
%
%\noindent
%{\bf Proof}. We use the notation:
%\[
%P_v(i) \;=\;\frac{N_v^i}{N},\;\;\;i \neq 0;\;\;\;
%P_{v0}\;\equiv\;P_v(0)\; = \;1 - \sum_{i\neq 0}P_v(i).
%\]
%These values represent the (empirical) probabilities of $v$
%taking value $i\neq 0$ and 0 respectively.  Entropies will be denoted
%by $H$. We aim to show that $\fracpartial{I_{uv}}{P_{v0}} < 0$....\\
%
%{\noindent \em Remainder omitted in this sample. See http://www.jmlr.org/papers/ for full paper.}
%\section{Section title of first appendix}\label{secA1}
%
%An appendix contains supplementary information that is not an essential part of the text itself but which may be helpful in providing a more comprehensive understanding of the research problem or it is information that is too cumbersome to be included in the body of the paper.

%%=============================================%%
%% For submissions to Nature Portfolio Journals %%
%% please use the heading ``Extended Data''.   %%
%%=============================================%%

%%=============================================================%%
%% Sample for another appendix section			       %%
%%=============================================================%%

%% \section{Example of another appendix section}\label{secA2}%
%% Appendices may be used for helpful, supporting or essential material that would otherwise 
%% clutter, break up or be distracting to the text. Appendices can consist of sections, figures, 
%% tables and equations etc.
% }{} 

\end{appendices}

%%===========================================================================================%%
%% If you are submitting to one of the Nature Portfolio journals, using the eJP submission   %%
%% system, please include the references within the manuscript file itself. You may do this  %%
%% by copying the reference list from your .bbl file, paste it into the main manuscript .tex %%
%% file, and delete the associated \verb+\bibliography+ commands.                            %%
%%===========================================================================================%%

%\bibliography{sn-bibliography}% common bib file
\bibliography{paper}% common bib file
}{
}
%% if required, the content of .bbl file can be included here once bbl is generated
%%\input sn-article.bbl

\end{document}
